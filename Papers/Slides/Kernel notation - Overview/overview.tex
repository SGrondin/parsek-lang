\documentclass[utf8]{beamer}

\usetheme{split}
\setbeamercovered{dynamic}

\newcommand{\out}[1]{}
\newcommand{\df}[1]{\stackrel{\mathtt{def}}{#1}}
\newcommand{\fleche}[1]{\displaystyle \mathop {\mbox{\rightarrowfill}}^{#1} }
\newcommand{\Kernel}{\textit{Kernel}}


%\newcommand{\mypause}{}
\newcommand{\mypause}{\pause}

\title[\Kernel{} Notation]{\Kernel{}: functional notation for system description}

\author{Wojciech Fraczak}

\institute{ 
  Dépt d'informatique, Université du Québec en Outaouais, 
  Gatineau PQ, Canada 
}

\date[]{}

\begin{document}
\begin{frame}
  \maketitle
\end{frame}


\section{Introduction}
\subsection{Introduction}
\begin{frame}
  \frametitle{Introduction}
  \begin{itemize}
  \item \Kernel{} is a notation for describing computer systems \mypause
  \item \Kernel{} is both a programing language which can be compiled
    into a highly optimized machine code, and a specification language
    for modeling and verification purposes \mypause
  \item Main features unique to \Kernel{} are:
    \begin{itemize}
    \item the uniform and minimalistic syntax; 
    \item precise mathematical semantics in terms of relations;
    \item execution model (virtual machine) in terms of FIFO system.
%    \item composition instead of ``function call'';
%    \item interchangeable nature of behavioral and functional aspects of
%      system components.
    \end{itemize}
  \end{itemize}
\end{frame}

\subsection{History and motivation}

\begin{frame}
  \frametitle{History and motivation}
  \begin{itemize}
  \item[1] My Solidum/IDT Canada experience: PDL/PTL (Packet Description/Transformation Language)
    \begin{itemize}
    \item for programming one-state push-down automaton implemented in hardware
    \end{itemize} \mypause
  \item[2] Edgewater: RTEdge --- a real-time programming language
    with integrated verifier for ``meeting deadlines''
    \mypause
  \item In both cases the proprietary languages were needed even
    though hundreds of programming languages had already existed...
    \mypause
  \item \Kernel{} aims to become a common notation easily extensible
    ... to meet PTL or RTEdge requirements, for example.
  \end{itemize}
\end{frame}

\subsection{What Kernel is for...}

\begin{frame}
  \frametitle{What \Kernel{} is not...}
  \begin{itemize}
  \item \Kernel{} is not a full featured general purpose programming
    language for IT professionals ...

    {\small Even though one could use \Kernel{} to develop a Web site or a
    flashy graphical user interface, the notation was not designed for
    it ... }

\mypause
  \item You may be interested in \Kernel{} if you want your program to
    be \textcolor{red}{\bf provably}:
    \begin{itemize}
    \item fast,
    \item resource efficient, and/or
    \item correct.
    \end{itemize}

  \end{itemize}
\end{frame}

\section{Overview of the approach}

\begin{frame} 
  \frametitle{Overview of the approach}
  \begin{center}
    \input{schema.pstex_t}
  \end{center}
\end{frame}

\section{Outline of the talk}

\begin{frame} 
  \frametitle{Outline of the talk}
  \begin{enumerate}
  \item Short introduction to \Kernel{} notation
  \item How do we compile a \Kernel{} program into FIFO system 
  \item Verification by \emph{over-approximations} 
  \end{enumerate}

  \hfill \input{schema-talk.pstex_t}
\end{frame}

\section{Introduction to Kernel notation}

\subsection{Values as rooted trees}
\begin{frame}[fragile]
  \frametitle{Flat (printable) types, values, and patterns}
    \textcolor{red}{\texttt{<type> nat <is> ['0 \{\}, '1 nat ];}}
  \begin{center}
    \input{nat.pstex_t}      
  \end{center} \mypause 
{\footnotesize Values: \verb+'0+,~ \verb+'1'0+,~
  \verb+'1'1'0+,~ \verb+'1'1'1'0+, ...\mypause, are \emph{trees}!}\\[0.4cm]

  \mypause 

  A \Kernel{} type is:
  \begin{itemize}
  \item a set of values (finite tree automata = and-or graphs),
  \item a set of constructors (field names), and
  \item a set of \emph{patterns} (initial parts of values).
  \end{itemize}
\end{frame}

\subsection{Relations}
\begin{frame}
  \frametitle{\Kernel{} components are \emph{relations}}
  \begin{center}
    \input{relation.pstex_t}
  \end{center}
  Every non-recursive \Kernel{} component can be translated into one
  (possibly very big) state machine.
\end{frame}

\subsection{Syntax}

\begin{frame}[fragile]
\frametitle{\Kernel{} syntax by example}
\footnotesize \mypause
\verb+<type> nat <is> ['0, '1 nat ];       -- (recursive) variant type+\\[-0.06cm] \mypause
\verb+<type> pair <is> {'x nat, 'y nat};   -- record type+\\[0.1cm]\mypause
\verb+(pair -> pair) swap = (pair -> {'x $.y, 'y $.x});+ \\[0.1cm] \mypause
\verb+({} -> nat) get_nat;                 -- FIFO+\\[0.1cm]\mypause
\verb+({} -> pair) get_pair = {'x get_nat, 'y get_nat} <by> CONCAT; +\\[0.1cm] \mypause
\verb+(pair -> nat) x-y = +\\[-0.06cm]
\verb+   [ ({'x ..., 'y '0 } -> $.x), +\\[-0.06cm]
\verb+     ({'x '1 ..., 'y ... } -> {'x $.x.1, 'y $.y.1}) :: x-y +\\[-0.06cm]
\verb+   ] <by> FIRST_MATCH; + \\[0.1cm]\mypause
\verb+(pair -> nat) y-x = swap :: x-y; +\\[0.1cm] \mypause
\verb+({} -> nat) main = +\\[-0.06cm]
\verb+ [ {'y get_nat, 'x '1 @y} :: x-y, +\\[-0.06cm]
\verb+   get_nat, +\\[-0.06cm]
\verb+   get_pair :: x-y  +\\[-0.06cm]
\verb+   <longest> /* <shortest>, <unknown> */ get_nat +\\[-0.1cm]
\verb+ ]; +
\end{frame}


\section{Compilation}



\subsection{Building state machines} 
\begin{frame}
  \frametitle{Building state machines}
  Every construction in \Kernel{} is an operation of \emph{relations}.
  \begin{itemize}
  \item \textbf{Pattern matching} \mypause
    \begin{itemize}
    \item Every pattern corresponds to a regular tree language encoded
      by a ``simple language'' (one state push-down automaton).
    \item Regular tree languages are closed by union, intersection, and complement.
    \end{itemize} 
    \mypause
  \item \textbf{Product}, \textbf{Union}, and \textbf{Composition}
    \mypause are variations of composition of relations - realized by
    corresponding state products
    \mypause
  \item \textbf{Projection}  is relabeling
  \item ...
  \end{itemize}
\end{frame}

\begin{frame}
 \frametitle{An example...}
\begin{itemize}
\item Pattern \texttt{\{'x ..., 'y '0\}} of type \texttt{pair}:
  \[(x/x)(1/1)^{\ast}(0/0)(y/y)(0/0)\] \mypause

\item Projection $\texttt{\$.x}$ over argument of type \texttt{pair}:
  \[(x/)(1/1)^{\ast}(0/0)(y/)(1/)^{\ast}(0/)\] \mypause

\item Abstraction:
\[
  \underbrace{(\underbrace{\{\texttt{'x ..., 'y '0}\}}_{(x/x)(1/1)^{\ast}(0/0)(y/y)(0/0)} \rightarrow 
   \underbrace{\texttt{\$.x}}_{(x/)(1/1)^{\ast}(0/0)(y/)(1/)^{\ast}(0/)})}_{(x/)(1/1)^{\ast}(0/0)(y/)(0/)}
\]
\end{itemize}
\end{frame}

\subsection{FIFO system}
\begin{frame}
  \frametitle{FIFO system}
  \begin{itemize}
  \item A FIFO system is a network of finite state machines connected
    by point-to-point FIFO channels. \mypause
  \item Since FIFOs are unbounded, even a single state machine with a
    couple of FIFOs is Turing Machine equivalent. \mypause
  \item Advantages of FIFO virtual machine representation are:
    \begin{itemize}
    \item simple and easy to implement \mypause 
    \item admits distributed implementation \mypause
    \item friendly to formal verification via over-approximation
    \end{itemize}

  \end{itemize}
\end{frame}

\subsection{Connecting state machines into FIFO system}
\begin{frame}
  \frametitle{Connecting state machines into FIFO system}
  \begin{itemize}
  \item Finiteness of state representation of relations is not always
     preserved, e.g., because of recursion. \mypause
   \item We overcome the state explosion problem by constructing a
     FIFO system instead of a single transducer. \mypause
   \item The challenge is in constructing a ``useful'' FIFO system  
     \mypause
     \vfill
   \item[] \textcolor{blue}{many different techniques may be
       considered to build a FIFO system from a \Kernel{}
       specification... this remains our main research topic... }
  \end{itemize}

\end{frame}


\section{Verification}
\subsection{Over-approximation}
\begin{frame}
  \frametitle{Over-approximation}
  \begin{itemize}
  \item In theory, a finite state FIFO system (i.e., with bounded size
    channels) can be verified using SPIN/Promela tool set ... \mypause
    --- however, checking if FIFO system is ``finite state'' is
    undecidable! \mypause
  \item Over-approximation:
    \begin{itemize}
    \item over-approximation of $S$ consists in adding ``executions''
      to the system creating its ``over-approximation'' $S'$,
      hopefully simpler to analyze \mypause
    \item to prove that every execution of $S$ has a property $\phi$,
      we prove that  every execution of $S'$ has the property $\phi$ \mypause
    \item E.g., the over-approximation can be used in proving that a
      WCRT is bounded by a ``deadline''
    \end{itemize}
  \end{itemize}

\end{frame}

\subsection{Over-approximation techniques}
\begin{frame}
  \frametitle{Over-approximation techniques}
  \begin{itemize}
  \item We have considered two over-approximations of a FIFO system
    $S$: \mypause
    \begin{itemize}
    \item Transformation of $S$ into a finite state machine: \mypause

      It can be done by a successive elimination of FIFO channels. \mypause

      \textcolor{blue}{Question: How a FIFO channel can be eliminated?}

      \mypause
    \item Transformation of $S$ into a Petri Net. \mypause
      
      If the FIFO channels are replaced by ``any-order'' containers
      then without loosing any possible execution of the initial FIFO
      system, we can replace every channel by a finite set of Places.
    \end{itemize}
  \end{itemize}
\end{frame}

\begin{frame}
  \frametitle{Conclusions}
  \begin{itemize}
  \item We propose a simple but powerful notation for describing
    computing systems \mypause
  \item Semantics of the notation is expressed in terms of
    multi-dimensional ``relations'' which can be effectively
    implemented as a FIFO system \mypause
  \item We mentioned two techniques which can transform FIFO system by
    over-approximation into:
    \begin{itemize}
    \item a finite state machine
    \item a Petri-Net
    \end{itemize} 
    \mypause
  \item This is work in progress ... \textcolor{blue}{especially
      meaningful FIFO system construction}
\end{itemize}
\end{frame}

\begin{frame}
\frametitle{Thank you}

Thank you! \mypause
\end{frame}
the end 
\end{document}

