\documentclass{beamer}
\let\Tiny=\tiny
\usetheme{split}
\usepackage[frenchb]{babel}
\usepackage[utf8x]{inputenc}
\usepackage{synttree}
%\setbeamercovered{dynamic}

\newcommand{\out}[1]{}
\newcommand{\df}[1]{\stackrel{\mathtt{def}}{#1}}

\newcommand{\mypause}{\pause}
%\newcommand{\mypause}{}

\newcommand{\fleche}[1]{\displaystyle \mathop {\mbox{\rightarrowfill}}^{#1} }
\newcommand{\Kernel}{\textit{Kernel}}


\title[INF4083: Kernel/CORE]{Langages de programmation (INF4083) \\
  Langages fonctionnels: Kernel/CORE} 

\author{Wojciech Fraczak}

\date{15 mars 2011}

\begin{document}
\begin{frame}
  \maketitle
\end{frame}


\section{Introduction}
\subsection{Introduction}
\begin{frame}
  \frametitle{Introduction}
  \begin{itemize}
  \item \Kernel{} --- une notation pour la description de systèmes
    informatiques --- \textcolor{blue}{mon projet de recherche!} \mypause
  \item \Kernel{} est un langage de programmation qui peut être compilé
    en code machine, et un langage de  spécification pour la
    modélisation et la vérification. \mypause
  \item Les traits uniques de \Kernel{} sont:
    \begin{itemize}
%    \item une syntaxe simple et uniforme; 
    \item une sémantique en termes de \alert{\emph{relations}} (pas fonctions totales!);
    \item \alert{\emph{composition}} à la place de l'\emph{application};
    \item modèle (machine virtuelle) en  termes de \alert{systèmes FIFO};
    \item pas de librairies prédéfinies et pas de ``built-in types''. 
%    \item interchangeable nature of behavioral and functional aspects of
%      system components.
    \end{itemize}
  \end{itemize}
\end{frame}

\subsection{Motivation}

\begin{frame}
  \frametitle{Motivation}
  \begin{itemize}
  \item[1] Mon expérience à Solidum/IDT Canada: PDL/PTL (Packet Description/Transformation Language)
    \begin{itemize}
    \item pour la programmation d'un automate à pile, un CPU/NPU
    \end{itemize} \mypause
  \item[2] Edgewater: RTEdge --- langage temps réel pour faciliter la vérification des contraintes temporelles 
    \mypause
  \item Dans les deux cas des nouveaux langages étaient nécessaires
    même si des milliers de langages de programmation existent déjà...
    \mypause
  \item \Kernel{} se donne comme objectif d'être une base (noyau) qui
    puisse être facilement étendue pour satisfaire les exigences de
    PTL, RTEdge, et autres langages dédiés.
  \end{itemize}
\end{frame}

\subsection{À quoi ça sert ...}

\begin{frame}
  \frametitle{À quoi ça sert ...}
  \begin{itemize}
  \item \Kernel{} n'est pas un langage de programmation d'usage
    générale pour les professionnels de l'informatique...

    {\small\color{gray} ...même si l'on pouvait l'utiliser pour
      développer des cites WEB ou pour des GUIs, la notation \Kernel{}
      n'était pas conçue pour ça.}


\mypause
\item On peut être intéressé par \Kernel{} si on cherche à développer
  des programmes avec des \alert{preuves} que: 
    \begin{itemize}
    \item le programme utilise des ressources d'une manière contrôlée,
    \item le programme est \alert{correct}.
    \end{itemize}

  \end{itemize}
\end{frame}

\section{Survole de l'approche}

\begin{frame} 
  \frametitle{Survole de l'approche}
  \begin{center}
    \input{schema.pstex_t}
  \end{center}
\end{frame}

\section{Plan du cours}

\begin{frame} 
  \frametitle{Plan du cours}
  \begin{enumerate}
  \item Introduction à la notation \Kernel{} 
  \item Compilation vers le système FIFO 
  \item Vérification par \emph{over-approximations} 
  \end{enumerate}

  \hfill \input{schema-talk.pstex_t}
\end{frame}




\section{Syntaxe}

\subsection{Types de données}

\begin{frame}[fragile]
  \frametitle{Types de données:}

  En BNF:
  {\footnotesize\color{red}
\begin{verbatim}
  Type_def     ::=  "<type>" Type_Name "<is>" Type ";"
  Type         ::=  Record_Type | Variant_Type | Type_Name
  Record_Type  ::=  "{" Label Type "," ... "}"
  Variant_Type ::=  "[" Label Type "," ... "]"
  Label        ::=  "'" Identifier 
  Type_Name    ::=  Identifier
\end{verbatim} 
  }
    
  Par exemple:{\footnotesize
    \begin{itemize}
    \item[] \color{blue}\verb+<type> UNIT <is> {}; +
    \item[] \color{blue}\verb+<type> BOOL <is> [ 'true {}, 'false {} ]; +
    \end{itemize}
  }
  
  Par convention, on peut éliminer les accolades \verb+{}+. Par exemple:
  \begin{quote} \footnotesize
    \color{blue}\verb+<type> BOOL <is> [ 'true, 'false ]; +
  \end{quote}
  
  
\end{frame}

\begin{frame}[fragile]
  \frametitle{Types de données, valeurs et patterns}\small
  \textcolor{blue}{\texttt{<type> nat <is> ['0 \{\}, '1 nat ];}}
  \begin{center}
    \input{nat.pstex_t}      
  \end{center} \mypause 
  {\footnotesize Valeurs possibles: {\color{blue}\verb+'0+},~ {\color{blue}\verb+'1'0+},~
    {\color{blue}\verb+'1'1'0+}, etc.\mypause, sont des \alert{arbres}!}\\[0.4cm]
  
  \mypause 
  
  Dans \Kernel{}, un \og\alert{type}\fg{} est:\footnotesize
  \begin{itemize}
  \item un ensemble de valeurs (arbres),
  \item un ensemble de \emph{constructeurs} (attributs), et
  \item un ensemble de \emph{patterns} (la partie initiale de la valeur, e.g., \og{}{\color{blue}\verb+'1 ...+}\fg{}).
  \end{itemize} ~
\end{frame}

\begin{frame}[fragile]
  \frametitle{Types de données, valeurs et patterns: un exemple}
  {\footnotesize\color{blue}
\begin{verbatim}
      <type> bit <is> ['0, '1];
      <type> 2bits <is> {'b0 bit, 'b1 bit};
      <type> 2bits-bis <is> {'b1 ['0,'1], 'b0 ['1,'0]};
\end{verbatim} 
  }\small
  \begin{itemize}
  \item Valeurs: \mypause 
{\color{blue}\verb+{'b0 '0, 'b1 '0}+}, \mypause 
{\color{blue}\verb+{'b0 '0, 'b1 '1}+}, \mypause
{\color{blue}\verb+{'b0 '1, 'b1 '0}+}, \mypause
{\color{blue}\verb+{'b1 '1, 'b0 '1}+}. 

\mypause
\footnotesize
\item Pattern \og{}{\color{blue}\verb+2bits+}\fg{} (ou {\color{blue}\verb+{'b0 bit, 'b1 bit}+}, etc) 
  \mypause
  dénote toutes les valeurs de ce type!
  \mypause
\item Pattern 
  \og{}{\color{blue}\verb+{'b0 '0, 'b1 bit}+}\fg{}\mypause, dénote
  {\color{blue}\verb+{'b0 '0, 'b1 '0}+} et
  {\color{blue}\verb+{'b0 '0, 'b1 '1}+}.
\mypause
\item Si le type de pattern est non ambiguë, on peut utiliser \og{\color{blue}\verb+...+}\fg, e.g.: 
 \og{\color{blue}\verb+{'b0 '0, 'b1 ...}+}\fg, ou même  \og{\color{blue}\verb+{'b0 '0, ...}+}\fg.
\end{itemize}


\end{frame}


\subsection{Relations}
\begin{frame}
  \frametitle{Les composantes de \Kernel{} sont des \emph{relations}}
  \begin{center}
    \input{relation.pstex_t}
  \end{center}
  Chaque composante (non-récursive) peut être transformée en un automate
  fini (à plusieurs bandes).
\end{frame}


\begin{frame}[fragile]
  \frametitle{Intuition}
  \begin{itemize}
  \item Produit (parallélisme): \og{}{\color{blue}\verb+{'one R1, 'two R2}+}\fg{}

    \begin{quote}\footnotesize
      Si {\color{blue}\verb+R1:({} -> T1)+} et  {\color{blue}\verb+R2:({} -> T2)+}, alors   
      {\color{blue}\verb+{'one R1, 'two R2}:({} -> {'one T1, 'two T2})+}.
    \end{quote}
  \item Union (concurrence):   \og{}{\color{blue}\verb+[R1, R2]+}\fg{}

    \begin{quote}\footnotesize
      Les relations {\color{blue}\verb+R1+}, {\color{blue}\verb+R2+}, et
      {\color{blue}\verb+[R1, R2]+} sont tous de même type.
    \end{quote}

  \item Abstraction: \og{}{\color{blue}\verb+( $x Pattern -> R )+}\fg{}

    \begin{quote}\footnotesize
      Si {\color{blue}\verb+R:({} -> T1)+} et
      {\color{blue}\verb+Pattern:T2+}, alors
      {\color{blue}\verb+($x Pattern -> R):(T2 -> T1)+}.
    \end{quote}
   \item Composition (piping): \og{}{\color{blue}\verb+R1 :: R2+}\fg{}

    \begin{quote}\footnotesize
      Si {\color{blue}\verb+R1 : (T1 -> T3)+} et
      {\color{blue}\verb+R2 : (T3 -> T2)+}, alors
      {\color{blue}\verb+R1::R2 : (T1 -> T2)+}.
    \end{quote}
  
  \end{itemize}
\end{frame}

\begin{frame}[fragile]
\frametitle{Syntaxe}

En BNF: {\scriptsize\color{red}
\begin{verbatim}
Rel_def    ::= "(" Type "->" Type ")" Rel_Name "=" Expr ";"
             | "(" "{}" "->" Type ")" Rel_Name "=" "<input>" ";"
             | "(" Type "->" "{}" ")" Rel_Name "=" "<output>" ";"
Expr       ::= Rel_Name | "$" Var_Name | "@" Field_Name   
             | Label Expr  |  Expr "." Label  |  Expr "::" Expr 
             | "(" "$" Var_Name Pattern "->" Expr ")" 
             | "{" Label Expr "," ... Serialize "}" 
             | "[" Expr "," ... Priority "]"
Pattern    ::= "..." | Type | Label Pattern 
             | "{" Label Pattern "," ... "}"
Serialize  ::= | "<serialize>" Rel_Name Serialize
Priority   ::= | "<unknown>" | "<longest>"  Rel_Name Priority
             | "<shortest>" Rel_Name Priority
\end{verbatim}}


\end{frame}


\begin{frame}[fragile]
\frametitle{Syntaxe par exemple}\color{blue}
\footnotesize \mypause
\begin{verbatim}
<type> bool <is> ['true {}, 'false {}];

({} -> bool) t = 'true {}; 

(bool -> bool) not =
      [ ($x 'true -> 'false {}),    
        ($x 'false -> 'true {}) ];

({'x bool, 'y bool} -> bool ) or = 
   [ ($x {'x 'true,...} -> 'true {}),
     ($x {'x 'false,...} -> $x.y) ];    	

({} -> bool) main = {'x t, 'y t :: not} :: or;
\end{verbatim}
\end{frame}

\begin{frame}[fragile]
\frametitle{Syntaxe par exemple (notation simplifiée)}\color{blue}
\footnotesize \mypause
\begin{verbatim}
<type> bool <is> ['true, 'false];  

bool t = 'true;                    

(bool -> bool) not =
      [ ('true -> 'false),         
        ('false -> 'true) ];       

({'x bool, 'y bool} -> bool ) or = 
   [ ({'x 'true,...} -> 'true),    
     ({'x 'false,...} -> $.y) ];   	

bool main = {'x t, 'y t :: not} :: or; 
\end{verbatim}
\end{frame}

\begin{frame}[fragile]
\frametitle{Syntaxe par exemple (notation simplifiée)}\color{blue}
\footnotesize 
\begin{verbatim}
<type> bool <is> ['true, 'false];  -- ['true {}, 'false {}];

bool t = 'true;                    -- ({} -> bool) t = 'true {}; 

(bool -> bool) not =
      [ ('true -> 'false),         -- ($x 'true -> 'false {})    
        ('false -> 'true) ];       -- ($x 'false -> 'true {})

({'x bool, 'y bool} -> bool ) or = 
   [ ({'x 'true,...} -> 'true),    -- ($x {'x 'true,...} -> 'true {})
     ({'x 'false,...} -> $.y) ];   -- ($x {'x 'false,...} -> $x.y)   	

bool main = {'x t, 'y t :: not} :: or;  -- ({} -> bool) main = ...
\end{verbatim}
\end{frame}


\section{Compilation}



\subsection{Construire l'automate} 
\begin{frame}
  \frametitle{Construire l'automate}
  Chaque construction de \Kernel{} est une opération sur les \emph{relations}.
  \begin{itemize}
  \item \textbf{Pattern} \mypause
    \begin{itemize}
    \item À chaque \emph{pattern} correspond un \alert{langage régulier d'arbres}
    \item Les langages réguliers d'arbres forme une algèbre de Boole:
      union, intersection, et complément des langages réguliers
      d'arbres restent réguliers.
    \end{itemize} 
    \mypause
  \item \textbf{Produit}, \textbf{Union}, et \textbf{Composition}
    \mypause sont des variations de compositions des relations.

    \mypause Ils correspondent aux opérations sur des automates.
  \item ...
  \end{itemize}
\end{frame}

\begin{frame}
 \frametitle{Un exemple...}
\begin{itemize}
\item Pattern \texttt{\{'x ..., 'y '0\}} de type \texttt{\{'x nat, 'y nat\}}:
  \[(x/x)(1/1)^{\ast}(0/0)(y/y)(0/0)\] \mypause

\item Projection $\texttt{\$.x}$:
  \[(x/)(1/1)^{\ast}(0/0)(y/)(1/)^{\ast}(0/)\] \mypause

\item Abstraction:
\[
  \underbrace{(\underbrace{\{\texttt{'x ..., 'y '0}\}}_{(x/x)(1/1)^{\ast}(0/0)(y/y)(0/0)} \rightarrow 
   \underbrace{\texttt{\$.x}}_{(x/)(1/1)^{\ast}(0/0)(y/)(1/)^{\ast}(0/)})}_{(x/)(1/1)^{\ast}(0/0)(y/)(0/)}
\]
\end{itemize}
\end{frame}

\subsection{Systèmes FIFO}
\begin{frame}
  \frametitle{Systèmes FIFO}
  \begin{itemize}
  \item Un système FIFO est un réseau d'automates finis connectés par
    canaux unidirectionnels point-à-point de type FIFO
    (First-In-First-Out). \mypause

  \item Les  avantages d'une telle représentation:
    \begin{itemize}
    \item simple et facile à réaliser \mypause 
    \item admet une réalisation distribuée \mypause
    \item facilite les techniques formelles de vérification via ``over-approximation''
    \end{itemize}

  \end{itemize}
\end{frame}

\subsection{Construction de  systèmes FIFO}
\begin{frame}
  \frametitle{Construction de systèmes FIFO}
  \begin{itemize}
  \item La finitude d'automates n'est pas toujours préservée, par
    exemple, à cause de la récursion \mypause
  \item On peut réduire l'augmentation importante de la taille en
    construisant un système FIFO à la place d'un seul automate
    \mypause
   \item Le défi est la construction d'un ``bon'' système  FIFO
     \mypause
     \vfill
   \item[] \textcolor{blue}{plusieurs techniques peuvent être
       envisagées pour la construction d'un système FIFO à partir d'un
       programme \Kernel{} ... ça reste un de mes projets de recherche
       ... }
  \end{itemize}

\end{frame}


\section{Vérification}
\subsection{Sur-approximation}
\begin{frame}
  \frametitle{Sur-approximation (over-approximation)}
  \begin{itemize}
  \item En théorie tout système FIFO (avec la taille de canaux finie)
    peut-être vérifié en utilisant SPIN/Promela ... \mypause --- mais
    déjà, la vérification si système FIFO est fini est indécidable
    \mypause
  \item Sur-approximation (Over-approximation):
    \begin{itemize}
    \item Une sur-approximation de système $S$ consiste à ajouter des
      ``exécutions'' au système pour créer sa ``sur-approximation''
      $S'$ plus simple pour analyser \mypause
    \item pour prouver que toute exécution de $S$ a une propriété
      $\phi$, on prouve que toute exécution de $S'$ vérifie $\phi$ \mypause
    \item Par exemple, l'over-approximation peut être utilisée pour
      borner le ``WCRT'' par son ``deadline'' (la date d'échéance).
    \end{itemize}
  \end{itemize}

\end{frame}



\begin{frame}
\frametitle{Fin}

Merci \mypause
\end{frame}
the end 
\end{document}

