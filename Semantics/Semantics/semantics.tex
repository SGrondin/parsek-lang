% BIBTEX REFERENCE NAME OF THIS DOCUMENT: \cite{solidum-???}

\documentclass[12pt]{article}
%%% A hack to produce the CVS revision string, e.g., `Revision 1.10'
%%% DO NOT EDIT THE FOLLOWING LINE!
\newcommand{\Revision}{$\makebox[0pt]{}$Revision: 1.13 $\makebox[-0.7ex]{}$} 

\usepackage{amsmath}
\usepackage{amsthm}
\usepackage{fullpage}
\usepackage{graphics}
\usepackage{color}
%\usepackage{times}

\newcommand{\out}[1]{} 
\newcommand{\status}[1]{} 
\newcommand{\df}[1]{\stackrel{\mathtt{def}}{#1}}

\newcommand{\sem}[1]{\left\|{#1}\right\|} 

\theoremstyle{plain}
\newtheorem{thm}{Theorem}
\newtheorem{prop}[thm]{Proposition}
\newtheorem{cor}[thm]{Corollary}
\newtheorem{lem}[thm]{Lemma}

\theoremstyle{definition}
\newtheorem{defn}{Definition}

\theoremstyle{remark}
\newtheorem{conj}{Conjecture}
\newtheorem{note}{Note}

\theoremstyle{example}
\newtheorem{ex}{Example}

\status{ ...draft... }
\title{Semantics of Kernel/CORE Language}

\author{Wojciech Fraczak}
\date{Universit\'e du Qu\'ebec en Outaouais  \\ 
  Gatineau, Qu\'ebec, Canada \\[1em] 
  \small\Revision}


\begin{document}
\maketitle

\begin{abstract}
  A semantics of Kernel/CORE Language is described.  In addition, we
  investigate a ``concrete semantics'' of Kernel Language which can be
  used as a starting point for design of a high-performance compiler.
\end{abstract}

\section{Introduction}

This paper is composed of two parts. In the first part we describe in
an abstract but precise way the main concepts of Kernel/CORE syntax and
semantics such as \emph{basic types}, \emph{templates},
\emph{environment}, and \emph{traces}. The second part defines what we
call ``concrete semantics''. Concrete semantics defines an
encoding of flat values and practical approach to the traces,
partitioning the trace alphabet into input and output.

\section{Abstract syntax}

Most often, the goal of a Kernel/CORE pogram is to define a
``\emph{relation}''. Such a program consists of basic type definitions
and relation definitions.  Kernel is ``flat'', which means that a
relation is a set of pairs of non-functional values (we call them
\emph{flat}, \emph{printable}, or \emph{first-order} values) of basic
types.

\subsection{Basic types}

All basic types are defined using two constructs: \emph{disjoint
  union} using square brackets and \emph{product} using curly
brackets.
\begin{verbatim}
  T ::=   [ 'l1 T1, ..., 'ln Tn ]  |  { 'l1 T1, ..., 'ln Tn }
\end{verbatim}
\emph{Unit type} corresponds to the empty product, i.e., $\{\}$; it
carries only one value also denoted by $\{\}$. \emph{Empty type} which
carries no value corresponds to the empty union, $[]$.

Names of union and product fields are called \emph{labels} and are
prefixed by a single quote. Within one union or one product definition
labels cannot repeat, however globally they can be reused. E.g.:
\begin{verbatim}
  <type> two_letter_alphabet <is> ['a {}, 'b {}];
  <type> three_letter_alphabet <is> ['a {},'b {}, 'c {}];
\end{verbatim}

Types can be named and recursive types can be defined. E.g.:

\begin{verbatim}
  <type> List_of_bits <is> [ 'empty {}, '0 List_of_bits, '1 List_of_bits ];
\end{verbatim}

Every basic type defines an enumerable set of \emph{flat values},
which are finite, edge labeled, rooted trees.  Basic types are
represented as vertices of a minimal labeled \emph{and-or} graph,
called \emph{type graph}: product types are presented by \emph{and}
vertices and union nodes by \emph{or} vertices.  The \emph{unit} type
is an \emph{and} vertex without children.  The type graph can be seen
as a deterministic top down tree automaton: the set of values of the
type corresponds to the set of trees accepted by the corresponding
tree automaton.


\subsection{Relations}

A relation $f$ (also called function since it generalizes the notion
of a function) is a set of ordered pairs of flat values, $f\subseteq
D_f\times I_f$, where $D_f$ and $I_f$ are two sets of values of some
basic types. Intuitively, $D_f$ denotes the domain of $f$ and $I_f$
the image of $f$. The pair of basic types $(t_D,t_I)$ corresponding to
$D_f$ and $I_f$ is called the \emph{type} of $f$ and denoted by
$f:T_D\rightarrow T_I$.  If for every $x\in D_f$, the set
$f(x)\df{=}\{y\in I_f\mid (x,y)\in f\}$ is empty or a singleton, then
$f$ defines a partial mapping from $D_f$ to $I_f$.

A relation $f$ from unit type, i.e., $f:\{\}\rightarrow I_f$, is
called a constant.  A named relation definition in Kernel looks like follows:
\begin{verbatim}
  (Tx -> Ty) rel_name = rel_body ; 
\end{verbatim}

The body of a relation is an expression \verb+E+ of the following
form:
\begin{verbatim}
 E  ::=  rel  |  @label  |  $var  |  'label E  |  E . label  
        |  product  |  union
        |  E :: E  |  ( $var P -> E )  
 product  ::=  { 'label E, 'label E, ... }  
 union  ::=  [ E, E, ... ]
 rel, label, var  ::=  identifier
\end{verbatim}
where \verb+@label+ is a reference to a member of product
constructor labeled by \verb+'label+, \verb+$var+ is a parameter name,
\verb+'label+ is a \emph{label}, and \verb+P+ is a
\emph{template}. The intuitive meaning of the above constructs are:
\begin{description}
\item[] \verb+rel+ --- reference to relation named \verb+rel+;
\item[] \verb+@label+  --- reference to a field of a product
  constructor;
\item[] \verb+$var+ --- reference to the actual parameter value;
\item[] \verb+'label E+ --- variant value constructor;
\item[] \verb+E . label+ ---  projection/selector;
\item[] \verb+{ 'label E, ... }+ ---  product value constructor; 
\item[] \verb+[ E, ... ]+ --- union
operation;
\item[] \verb+E :: E+ --- composition; and
\item[] \verb+($x P -> E)+ --- abstraction.
\end{description}

Intuitively, the product value constructor represents parallel
composition in which correponding field values are calculated by
evaluation of sub-expressions. \emph{Product annotations} can be added
to the expression (just before closing squarle backet) in order to
specify how the interaction of parallel sub-processes with environment
should be performed. Presently two strategies exist:
\begin{itemize}
\item empty annotation (i.e., default) --- in this mode the
  subprocesses interact with environment synchronously; and
\item \verb+<serialize> ChanName+ --- interaction of the subprocesses
  with environment is serialized on the pipe \verb+ChanName+; the
  order of the interaction in that channel is determined by the actual
  enumeration order of the fields in the product expression.
\end{itemize}

The union operation is defined as union of relations. In general,
union of functions yields a relation which is not a function. We
introduce \emph{union annotations} which flattens the relation
into a function (assumed all arguments are functions) by ordering
elements of the relation:
\begin{itemize}
\item empty annotation (default) --- only the first sub-expression
  which is defined is considered,
\item \verb+<longest> ChanName+ --- the sub-expression which is
  defined and which interacts the \emph{longest} on pipe
  ``\verb+ChanName+'' is considered, and
\item \verb+<shortest> ChanName+ --- the sub-expression which is
  defined and which interacts the \emph{shortest} on pipe
  ``\verb+ChanName+'' is considered.
\item \verb+<unknown>+ (must be the last annotation) explicitely
  states that the order is unknown (e.g., for
  verification/specification/abstraction purposes).
\end{itemize}

We require that every Kernel Language expression is of first order
type, and thus parameters have to be of a basic type, expressions
defining the body of a functional abstraction have to be a constant, and
all arguments for product constructors are constants.


\subsubsection{Sub-expressions and free references}

For every Kernel subexpresion $p$ we define the set of the free
references in the usual recursive way:
\begin{itemize}
\item If $p$ is a parameter name or a label of a product constructor,
  i.e., $p=\$x$ or $p=@l$, then $\texttt{free}(p)=\{p\}$.
\item If $p$ is a relation name, then
  $\texttt{free}(p)=\emptyset$, i.e., empty set.
\item If $p$ is a pipeline (composition) or a union
  operation, i.e., $p=p_1{::}p_2$ or $[p_1,
  p_2]$ then
  $\texttt{free}(p)=\texttt{free}(p_1)\cup\texttt{free}(p_2)$.

\item If $p$ is a product constructor, i.e., $\{f_1~ p_1, f_2~ p_2\}$,
  then $\texttt{free}(p) =
  (\texttt{free}(p_1)\cup\texttt{free}(p_2))\setminus\{@f_1,@f_2\}$.

\item If $p$ is a projection or a variant, i.e., $p=q.l$ or $p='l~q$,
  then $\texttt{free}(p)=\texttt{free}(q)$.
\item If $p$ is an abstraction, i.e., $p=(\$x ...\rightarrow q)$, then
  $\texttt{free}(p)=\texttt{q}\setminus \{\$x\}$.
\end{itemize}

A subexpression $p$ is called \emph{closed} if $\texttt{free}(p)$ is
empty.

Let $V$ be a countable set of \emph{variable names}, each typed by a
basic type. We call \emph{context}, any partial mapping from (a finite
part of) $V$ into values of corresponding types. E.g., if $V=\{x,y\}$
such that $x$ is of type $T_x$ and $y$ is of type $T_y$, then
$C:\{x\mapsto v_x, y\mapsto v_y\}$ is a context, with $v_x$ and $v_y$
being some values of types $T_x$ and $T_y$, respectively. The mapping
defined by a context naturally extends to Kernel expressions.


\subsubsection{Patterns (templates)}

Templates play the role of filters for values of the actual
parameters. Every flat type definition defines the set of values
(trees) together with a syntax for unary predicates on them, which we
call patterns. Intuitively, a pattern is a initial part of the tree
(from the root) which selects all values with the same initial part.

\begin{verbatim}
 P  ::=  T  | { 'l1 P1, ..., 'ln Pn } | 'label P 
\end{verbatim}
where \verb+T+ is a type expression, \verb+'l1+, \ldots, \verb+'ln+, 
\verb+'label+ are labels, and  \verb+P1+, \ldots, \verb+Pn+, 
\verb+P+ are patterns (templates).

The semantics of a pattern for a type is a non-empty subset of values
of the type.

\section{Semantics}

A Kernel program can describe functional and behavioral aspects of
computing. Even though we introduced some constructions (such as three
kinds of union and two kinds of product constructors) which explicitly
deal with behavioral aspect of computing, we have not provided any
primitives. 

We believe that Kernel can be used in various contexts in which the
primitives would be very different. Thus, we assume that Kernel is
used in a context for which the set of \emph{synchronization events}
is defined apart and it is accessible to a Kernel Language user
through a set of predefined relations and basic types.

More precisely, we assume the existence of \emph{Kernel environment},
$\mathcal{E}$ defined elsewhere, which consists of a set of
\emph{synchronization events} and a composition operation which will
allow us to build complex \emph{traces} (i.e., sequences of
synchronization events) starting from atomic traces consisting of a
single synchronization event. Such an environment can be formalized by
a notion of \emph{monoid}. A monoid is a triple
$\mathcal{M}=(M,\cdot,1)$, where $M$ is a set, $\cdot$ is an
associative total mapping $M\times M\mapsto M$, and $1$ is an elements
from $M$ called unit, which verifies $x\cdot 1= 1\cdot x =x$, for all
$x\in M$.  In a monoid $(M,\cdot,1)$, we define \emph{prefix
  relation}, $\leq$, as follows: $x\leq y$ if there exists $z\in M$
such that $xz=y$.

We will suppose that a monoid $\mathcal{E}=(M,\cdot,1)$ which
represents a Kernel environment verifies the two following conditions

\begin{itemize}
\item \emph{prefix relation} is an order, i.e., $x=xyz$ implies
  $x=xy$, for all $x,y,z\in M$; and
\item \emph{prefix relation} generates unique infimum, i.e., $x\vee
  y\df{=}\min\{z\in M\mid x\leq z, y\leq z\}$, if exists, is unique.
  If $x\vee y$ exists then we write $x\perp y$. Otherwise, i.e., when
  $x\vee y$ does not exist, we write $x\not\perp y$.
\end{itemize}

\begin{prop}\label{prop:environment}
  The following monoids are \emph{environment monoids}:
  \begin{itemize}
  \item any free monoid;
  \item any set monoid (i.e., idempotent commutative free monoid);
  \item if $M_1$ and $M_2$ are environment monoids then $M_1\times
    M_2$ is an environment monoid.
  \end{itemize}
\end{prop}

Proposition~\ref{prop:environment} allows us to consider the following
syntax for defining monoids which, by construction, will be
environment monoids:
%%
\[
M~::=~ \{e_1,\ldots,e_n\}^{\ast} ~\mid~ \mathcal{P}{\{e_1,\ldots,e_n\} } ~\mid~
M \times M
\]
%%
for any $n \geq 0$. Construction $\{e_1,\ldots,e_n\}^{\ast}$ defines a
free monoid generated by $n$ prime elements $e_1,\ldots,e_n$, and
construction $\mathcal{P}{\{e_1,\ldots,e_n\}}$ defines a set monoid
generated by $n$ prime elements $e_1,\ldots,e_n$.


From a practical point of view one can see the environment as a
collection of FIFO channels $C_1,C_2,\ldots,C_k$. In such a model,
every channel defines a free monoid of ``atomic data packets'' which
can be carried by the channel. The whole environment is the product
monoid of all those free monoids. In such a case, the modifiers of
union and product constructions can be selectively chosen per the
channel and the direction (read or write) basis (called also
``dimension'').

\subsection{Non recursive definitions}

With every Kernel subexpression $p$ of type $(T_a\mapsto T_b)$ we
associate its semantics, $\sem{p}$, i.e., a mapping from a context
defined on $\texttt{free}(p)$ to a set of quadruples $(m,n,a,b)$, where $m$
 and $n$ are elements of the environment monoid, $a$ is a flat value of type
$T_a$, and $b$ is a flat value of type $T_b$. Instead of writing
$\sem{p}(C)$ we will write $\sem{p}_C$. If $p$ is closed then
$\sem{p}_C$ does not depend on $C$ and thus we will write $\sem{p}$
for denoting the unique set of quadruples $(m,n,a,b)$ belonging to
$\sem{p}_C$.

Intuitively, a quadruple $(m,n,a,b)\in\sem{q}$ means, that $q$ can map
$a$ into $b$ in the environment context $m n$ by effectively consuming
$m$ (the $n$ part represents a ``look-ahead'').

\begin{description}

\item[unit:] $\sem{\{\}}_C \df{=} \{(1,n,\{\}, \{\})\mid n\in M\}$.
\item[identifier:] $\sem{\$x}_C \df{=} \{(1,n,\{\}, C(\$x)) \mid n\in M\}$.

\item[variant:] $\sem{f~q}_C \df{=} \{(m,n,a,f~b)\mid
  (m,n,a,b)\in\sem{q}_C\}$.

\item[projection:] $\sem{q.f}_C\df{=}\{(m,n,a,b.f)\mid
  (m,n,a,b)\in\sem{q}_C\}$.

\item[structure:] We distinguish two synchronization modes which
  yields two product value constructors (defined by ``dimension''):
  
  $\sem{\texttt{CONCAT}\{f_1~q_1,f_2~q_2\}}_C \df{=} \{(m_1m_2,
  m_2^{-1}(n_1\vee m_2n_2), \{\},\{f_1~b_1,f_2~b_2\}) \mid\\ ~~ 
  (m_1,n_1,\{\},b_1)\in\sem{q_1}_C,
  (m_2,n_2,\{\},b_2)\in\sem{q_2}_{C.[@f_1\mapsto b_1]} \}$
  
  $\sem{\texttt{INTER}\{f_1~q_1,f_2~q_2\}}_C \df{=} \{(m_1\vee
  m_2,(m_1\vee m_2)^{-1}(m_1n_1\vee m_2n_2), \{\},\{f_1~b_1,f_2~b_2\})
  \mid \\ ~~ (m_1,n_1,\{\},b_1)\in\sem{q_1}_C,
  (m_2,n_2,\{\},b_2)\in\sem{q_2}_{C.[@f_1\mapsto b_1]} \}$


\item[union:] Disambiguation can be done in three different ways.

  $\sem{\texttt{FIRST}[q_1,q_2]}_C \df{=}
  \sem{q_1}_C\cup \\
  \{(m,nn',a,b) \mid n'\in M, (m,n,a,b)\in\sem{q_2}_C,
  \forall (m_1,n_1,a,b_1)\in\sem{q_1}_C m_1n_1\not\perp mnn'\}$

  $\sem{\texttt{LONGEST}[q_1,q_2]}_C \df{=}\\
  \{(m,n,a,b)\in\sem{q_1}_C\mid 
  \forall (m',n',a,b')\in\sem{q_2}_C 
  (m'n'\perp mn \Rightarrow m\not\leq m')\}
  \cup \\
  \{(m,n,a,b)\in\sem{q_2}_C\mid 
  \forall (m',n',a,b')\in\sem{q_1}_C 
  (m'n'\perp mn \Rightarrow m'<m)\}
  $


  $\sem{\texttt{SHORTEST}[q_1,q_2]}_C \df{=}\\
  \{(m,n,a,b)\in\sem{q_1}_C\mid 
  \forall (m',n',a,b')\in\sem{q_2}_C 
  (m'n'\perp mn \Rightarrow m'\not\leq m)\}
  \cup \\
  \{(m,n,a,b)\in\sem{q_2}_C\mid 
  \forall (m',n',a,b')\in\sem{q_1}_C 
  (m'n'\perp mn \Rightarrow m<m')\} $


\item[composition:] 
  $\sem{q_1 :: q_2}_C \df{=} \\\{(m_1m_2,
  m_2^{-1}(n_1\vee m_2n_2), a, b) \mid
  (m_1,n_1,a,c)\in\sem{q_1}_C,(m_2,n_2,c,b)\in\sem{q_2}_C\}$

\item[abstraction:] $\sem{(\$x\,P \mapsto q)}_C\df{=} \{(m,n,a,b)\mid
  (m,n,\{\},b)\in\sem{q}_{C.[\$x\mapsto a]}\}$, where $C.[\$x\mapsto a]$
  denotes the following context:
  \[
  C.[\$x\mapsto a](v) \df{=}\left\{
    \begin{array}{ll}
      a & \mbox{ if } v = \$x \\
      C(v) & \mbox{ otherwise }
    \end{array}
  \right.
  \]
\end{description}

The above rules describe the semantics of all non-recursive Kernel 
definitions. 



We say that  $S=\sem{q}_C$ for a subexpression
$q:T_1\mapsto T_b$ is:
\begin{itemize}
\item \emph{functional}, whenever $(m,n,a,b),(m,n,a,b')\in S$ implies
  $b=b'$;
\item \emph{prefix-deterministic}, whenever $(m,n,a,b),(m',n',a,b')\in S$
  and $mn\perp m'n'$ implies $m=m'$ and $b=b'$;
\item \emph{finite}, whenever $\{(m,a,b)\mid (m,n,a,b)\in S\}$ is
  finite;
\item \emph{rational}, whenever $\{(m,a,b)\mid (m,n,a,b)\in S\}$ is
  rational;
\item \emph{simple}, ???
\item \emph{k-simple}, ???
\end{itemize}

\begin{prop}
  All above operations on $S$ preserve the above properties of its
  arguments. I.e., if $\sem{q_i}_C$ are \emph{functional} (resp.,
  prefix-deterministic, finite, rational), then variant, projection,
  concat, inter, first-match, longest-match, shortest-match,
  composition, and abstraction are \emph{functional} (resp.,
  prefix-deterministic, finite, rational).
\end{prop}

\begin{proof}
  No proof so far. It has to be proved!
\end{proof}


\subsubsection{Examples}

We consider the following environment:
\[
M ~ = ~ \{\texttt{get}[x] \mid x\in\texttt{BYTE}\}^{\ast}\times
        \{\texttt{put}[x] \mid x\in\texttt{BYTE}\}^{\ast}\times
        \{\texttt{lookup}[x,y] \mid x,y\in\texttt{BYTE}\}^{\ast}
\]
where $\texttt{BYTE}$ is a finite indexing set isomorphic to the set
of integers from $0$ to $255$. Thus, the primes (atomic actions) are:
$\texttt{get}[0],\ldots,\texttt{put}[255],\texttt{put}[0], \ldots,
\texttt{put}[255], \texttt{get}[0,0],\ldots,\texttt{put}[0,255],
texttt{put}[1,0], \ldots$. The above environment can be seen as two
FIFOs, one for reading and one for writing.

We assume the following Kernel declarations corresponding to the
primes:
\begin{verbatim}
   <type> BYTE <is> {'0, '1, '2, ..., '255};
   BYTE getByte = <input>;          // get
   (BYTE -> {}) putByte = <output>;  // put[]
\end{verbatim}
Notice that \texttt{getByte} and returns a value.
In order to keep a value produced by the prime, we put it as an index,
i.e., if a particular instance of $\texttt{getByte}$ returns value
$x$, then we denote it by $\texttt{get}_x$.

Let us consider the following Kernel function which reads three bytes and
outputs third byte first, then it outputs the result of the lookup on
the second byte, and finally outputs the first read byte.

\begin{verbatim}
type TRIPLE is {'a BYTE, 'b BYTE, 'c BYTE };
{} example1 =
   {'a getByte, 'b getByte, 'c getByte <serialize> getByte}
   :: ($x TRIPLE -> 
       $x.c :: putByte :: 
       $x.b :: lookup :: putByte ::
       $x.a :: putByte);
\end{verbatim}

The semantics of function \verb+example1+ is calculated bottom up
in the following way:
\begin{eqnarray*}
  \sem{getByte} & \df{=} & (\texttt{get}_x,\ast,\{\},x)\\
  \sem{putByte} & \df{=} & (\texttt{put}[x]_y,\ast,x,\{\})\\
  \sem{lookup} & \df{=} & (\texttt{lookup}[x]_y,\ast,x,y)\\
  \sem{\texttt{CONCAT}\{a~\texttt{getByte}, 
    \ldots, c~\texttt{getByte}\}}
  & = & 
  (\texttt{get}_i\texttt{get}_j\texttt{get}_k,\ast,\{\},\{a~i,b~j,c~k\})
  \\
  \sem{\$x.a}_C & = &  (1,\ast,\{\},C(x).a) \\
  \sem{\$x.c :: \texttt{putByte}}_C 
  & = & (\texttt{put}[C(x).c],\ast,\{\},\{\}) \\
  \sem{\$x.c :: \texttt{putByte} :: \$x.b :: \texttt{lookup}}_C 
  & = & (\texttt{put}[C(x).c]\texttt{lookup}[C(x).b]_y,\ast,\{\},y)\\
  \sem{\$x.c :: \ldots :: \texttt{putByte}}_C
  & = & 
  (\texttt{put}[C(x).c]
  \texttt{put}[y]
  \texttt{put}[C(x).a]
  \texttt{lookup}[C(x).b]_y,\ast,\{\},\{\})\\
  \sem{(\$x~\texttt{TRIPLE~->} \ldots)}
  & = & (\texttt{put}[x.c]
  \texttt{put}[y]
  \texttt{put}[x.a]
  \texttt{lookup}[x.b]_y,\ast,x,\{\})\\
  \sem{\texttt{example1}}
  & = &
  (\texttt{get}_i\texttt{get}_j\texttt{get}_k
  \texttt{put}[k]
  \texttt{put}[y]
  \texttt{put}[i]
  \texttt{lookup}[j]_y,\ast,\{\},\{\})\\
\end{eqnarray*}

The semantics of \texttt{example1} can also be depicted by the
following figure, with explicit causality.

\begin{center}
\input{order.pstex_t}
\end{center}

\begin{verbatim}
(TRIPLE -> BYTE) example2 = 
   ($x TRIPLE -> [
       $x.^a :: lookup,
       $x.^b :: lookup,
       $x.^c ]);
\end{verbatim}

The semantics of  function \verb+example2+ is calculated bottom up
in the following way (by $\overline{\texttt{lookup}[a]}$ we denote the
failure of action $\texttt{lookup}[a]$):
\begin{eqnarray*}
  \sem{\$x.a :: \texttt{lookup}}_C 
  & = & (\texttt{lookup}[C(x).a]_y,\ast,\{\},y)\\
  \sem{[\ldots, \$x.b :: \texttt{lookup}]}_C 
  & = & 
  (\texttt{lookup}[C(x).a]_y,\ast,\{\},y)~+ \\
  & &
  (\texttt{lookup}[C(x).b]_z,\overline{\texttt{lookup}[C(x).a]},\{\},z)
  \\
  \sem{[\ldots, \$x.c]}_C 
  & = & 
  (\texttt{lookup}[C(x).a]_y,\ast,\{\},y)~+ \\
  & &
  (\texttt{lookup}[C(x).b]_z,\overline{\texttt{lookup}[C(x).a]},\{\},z)~+
  \\
  & &
  (1,\overline{\texttt{lookup}[C(x).a]}\,
  \overline{\texttt{lookup}[C(x).b]},\{\},C(x).c)
  \\
  \sem{\texttt{example2}}
  & = &
  (\texttt{lookup}[x.a]_y,\ast,x,y)~+ \\
  & &
  (\texttt{lookup}[x.b]_z,\overline{\texttt{lookup}[x.a]},x,z)~+
  \\
  & &
  (1,\overline{\texttt{lookup}[x.a]}\,
  \overline{\texttt{lookup}[x.b]},x,x.c)
\end{eqnarray*}

Many graphical represenations (implementations) can be used to 
depict the semantics of \texttt{example2}, which consists of three
exclusive alternatives. For example:

\begin{center}
\input{choice.pstex_t}
\end{center}



\subsection{Recursive definitions}


Let $\vec{x}=(x_1, \ldots, x_n)$ be a vector of names of closed
recursive expressions of form $x_i=q_i(\vec{x})$, for $i\in[1,n]$.
The semantics $\vec{S}=(\sem{x_1},\ldots,\sem{x_n})$ is defined as a
fix point satisfying the defining equations. In general, a set of
equations may have none, one, or many solutions. 


A solution for a recursive definition can be calculated in the
following iterative way:

\begin{itemize}
\item $\vec{S}_0 = (\sem{x_1}_0,\ldots,\sem{x_n}_0) \df{=}
  (\emptyset,\ldots, \emptyset)$
\item $\vec{S}_i = (\sem{x_1}_i,\ldots,\sem{x_n}_i) \df{=} 
 \left(q_0(\vec{S}_{i-1}/\vec{x}),
  \ldots,q_n(\vec{S}_{i-1}/\vec{x})\right)$
\item $\vec{S}=(\sem{x_1},\ldots,\sem{x_n})$, where $\sem{x_i} \df{=}
  \{(m,n,a,b)\mid \exists k \forall j \geq k ~
  (m,n,a,b)\in\sem{x_i}_j\}$, for $i\in[1,n]$.
\end{itemize}

\section{Concrete semantics}

In the previous section we defined the abstract semantics of Kernel,
which makes possible reasoning about Kernel programs. In order to turn a
Kernel specification into a program which can be executed, we refine the
abstract semantics into \emph{concrete semantics}, where encoding
details of flat values and Kernel functions are described.

\subsection{Environment events separation}

In order to elaborate the concrete semantics, we separate the
synchronization events $\mathcal{E}$ into two groups: input events
$\mathcal{I}$ and output events $\mathcal{O}$, in such a way that
$\mathcal{E}=\mathcal{I}\times\mathcal{O}$. Intuitively, input events
consist of such events as reading some bits from the input packet or
querying a lookup table, and output events consist of writing to an
external (write-only) devices.

\subsection{Flat value encoding}

Given a type graph, we consider a very natural encoding of values of
types which the graph represents, by choosing an encoding alphabet
$\Sigma$, e.g., $\Sigma=\{0,1\}$, and labeling all outgoing edges of
each \emph{or} node by strings over the encoding alphabet in such a
way that no two different edges outcoming of the same node are labeled
by words which are in the prefix relation.

DFS of the hyper-path of a value defines its encoding...

\subsection{Graph representation of the semantics of a Kernel
  expression}

Let $\mathcal{E}=\mathcal{I}\times\mathcal{O}$ and $\Sigma$ be the
environment monoid and the flat value encoding alphabet, respectively.
We define \emph{semantics graph} $G=(V,E,s_0,s_f,F,\lambda)$ over
$\mathcal{M}=\Sigma^{\ast}\times\mathcal{E}\times\Sigma^{\ast}$, as a
digraph $(V,E)$, with one starting vertex $s_o\in V$, one ending vertex
$s_f\in V$ without outgoing edges, a set of final vertexes $F\subset V$,
and a edge labeling $\lambda:E\mapsto\mathcal{M}$.

Intuitively, vertices of the graph represent states and edges
represent transitions. An edge $t$ from vertex $s_1$ to $s_2$ and
labeled by $(w_i,e,w_o)\in\mathcal{M}$ tells us that the modeled
system with argument value encoded by $w_i$, when in state $s_1$ can
move to the state $s_2$ producing value $w_o$ and interacting with the
environment by $e$.

More precisely, every path $\pi$ from $s_0$ to $s_f$ passing by a
final vertex defines a quadruple $(m,n,a,b)$, where
$\lambda(\pi)=(w_a,mn,w_b)$ and $\lambda(\pi')=(w'_a,m,w'_b)$, where
$\pi'$ is the longest initial part of $\pi$ ending in a final vertex,
$w_a$ is the encoding of $a$ and $w_b$ is the encoding of $b$.



\subsubsection{Composition}

Given two finite semantics graphs $G_1$ and $G_2$ for expressions
$(T_1\rightarrow T_2) E_1$ and $(T_2\rightarrow T_3) E_2$, we can
construct a new finite semantics graph $G = G_1\circ G_2$ which
represents the functional composition $(T_1\rightarrow T_3) E_1::E_2$.

Without loss of generality we assume that all edge labels of $G_2$ are
of a form $(w_a,m,w_b)$ with $w_a$ being empty word or a letter, i.e.,
$w_a\in\Sigma\cup\{\varepsilon\}$. Therefore, a vertex $A$ in $G_2$
and a word $w\in\Sigma^{\ast}$ defines a set of vertexes $G_2(A).w$ of
$G_2$ which are reachable from $A$ by a path whose labels $w_a$
generate $w$. i.e., if $B\in G_2(A).w$ then there exists a path $\pi$
in $G_2$ starting in $A$ and such that $\label(\pi)=(w,m,w_b)$ for
some $m\in\mathcal{E}$ and $w_b\in\Sigma^{\ast}$.


The composition of $G_1$ and $G_2$ is obtained by a Cartesian product
of vertixes of $G_1$ and $G_2$, i.e., vertexes of $G$ as pairs
$(A,B)\in(V_1\times V_2)$.



\subsubsection{Projection}



\subsubsection{Structure}

\subsubsection{Union}

\subsubsection{Functional abstraction}

-----

In general, 



 
We require that such a graph be \emph{prefix-deterministic}, i.e., for
every two different edges $e_1$ and $e_2$ originating from the same
vertex $u$, we have
$\overline{\lambda(e_1)}\not\perp\overline{\lambda(e_2)}$.

Let $\mathcal{M}=M\times \Sigma^{\ast} \times \Sigma^{\ast}$, where
$M$ is an environment monoid and $\Sigma$ is a finite alphabet used
for encoding flat values. Notice, that $M\times \Sigma^{\ast}$ and
$\mathcal{M}$ are also environment monoids. Intuitively, an element
$(m,a,b)\in\mathcal{M}$ represents a step of calculation parametrized
by $a$, yielding $b$, and interacting with environment via $m$. By
$\overline{(m,a,b)}$ we denote $(m,a)$, i.e., projection onto two
first coordinates.

We define \emph{semantics graph} $G=(V,E,i,s,F,\lambda)$ over
$\mathcal{M}$, as a digraph $(V,E)$, with one starting vertex $i\in
V$, one ending vertex $s\in V$ without outgoing edges, a set of
finte vertexes $F\subset V$, and a edge labeling
$\lambda:E\mapsto\mathcal{M}$. We require that such a graph be
\emph{prefix-deterministic}, i.e., for every two different edges $e_1$
and $e_2$ originating from the same vertex $u$, we have
$\overline{\lambda(e_1)}\not\perp\overline{\lambda(e_2)}$.


\subsection{FIFO represenation of Kernel expressions}

Kernel program (without recursive constructs) can effectively be
translated into a network of transducers connected by FIFO
channels. In this section we describe a way of constructing the FIFO system.


Transducers (or more generally state based computing) are used to
encode \emph{functions} by \emph{behavior}.  We describe a
systematics method of encoding a Kernel program which contains functional
and behavioral description, into a transducer. In our presentation we
assume that the set of \emph{events} is finite and well defined.
Moreover, for the sake of this paper, the set of events is divided
into two disjoint classes:
\begin{itemize}
\item Read events: \texttt{getZero}, \texttt{getOne}, \texttt{getEof} 
\item Write events: \texttt{putZero}, \texttt{putOne}, \texttt{putEof} 
\end{itemize}


Alternatively, read events can be seen as elements of an input
alphabet and write events as elements of an output alphabet. For the
sake of simplicity we assume only the two FIFOs: \textit{I} and
\textit{O}. 

In order to implement all Kernel constructs we designed the generic
component interface for every Kernel sub-expression, see
Figure~\ref{fig:interface}.
%%
\begin{figure}[htb]
  \centering
  \input{interface.pstex_t}
  \caption{Kernel generic component interface.}
  \label{fig:interface}
\end{figure}
%%
The interface consists in:
\begin{itemize}
\item $A$ --- input FIFO through which the actual value of the argument
  is provided;
\item $R$ --- output FIFO through which the result (R) is communicated
  to the environment;
\item $I$, $O$ --- input and output FIFOs for environment monoid (in
  this presentation we limit ourselves to the one input and one output
  stream);
\item $I_d$, $I_c$ --- output FIFOS which splits the stream $I$ into
  what was effectively consumed (done), and the rest (continuation);
\item $O_c$ --- input FIFO for ``output continuation'';
\item $x_1,x_2,\ldots, x_n$ --- free variable FIFOS.
\end{itemize}

\subsubsection{Free variable references}

Reference to a field or an argument is just a FIFO system as depicted in
Figure~\ref{fig:reference}.
%%
\begin{figure}[htb]
  \centering
  \input{reference.pstex_t}
  \caption{FIFO element for a reference: $\$a$ or $@a$.}
  \label{fig:reference}
\end{figure}
%%
The ``eof'' box in the picture denotes a system which generates a
single ``end-of-file'' signal.

\subsubsection{Abstraction}

\begin{center}
\verb+($x P -> f)+
\end{center}

The pattern $P$ is considered as a filter over domain values of the
relation. The filter can be implemented by a CSM (i.e., using stack).

A stack can be implemented by a single state machine with one
self-looping FIFO channel. The interface of a stack consists of a
input channel caring the alphabet of the stack plus special symbol
\texttt{pop}, and an output channel caring the alphabet of the stack
plus special symbol \texttt{empty}. In Kernel it would be:

\begin{verbatim}
  <type> stack_alphabet <is> [ ... ];
  <type> request <is> ['push stack_alphabet, 'pop ];
  <type> answer <is> ['pop stack_alphabet, 'empty ];
\end{verbatim}

Therefore, a filter for every pattern can be implemented. The overall
module is depicted in Figure~\ref{fig:abstraction}.

\begin{figure}[htb]
  \centering
  \input{abstraction.pstex_t}
  \caption{Abstraction}
  \label{fig:abstraction}
\end{figure}

\subsubsection{Composition}
\begin{figure}[htb]
  \centering
  \input{composition.pstex_t}
  \caption{Composition}
  \label{fig:composition}
\end{figure}


\subsubsection{Product}
\begin{figure}[htb]
  \centering
  \input{product.pstex_t}
  \caption{Product by intersection}
  \label{fig:product}
\end{figure}



\subsubsection{Union}
\begin{figure}[htb]
  \centering
  \input{union.pstex_t}
  \caption{Union }
  \label{fig:product}
\end{figure}


\subsubsection{Variant constructor}
\subsubsection{Projection}




\subsection{Bottom-up transducer construction}

The write and read events are represented as built-in unit type
objects.  The semantics of an event is a two-state-one-transition
transducer, whose transition is labaled by the event. Alternatively, a
read event is represented as the two-state-one-transition transducer,
and a write event is represented as a one-state-no-transition
transducer with initial output being the write event. The name of an event
will be used to denote the corresponding transduser.

\subsection{Value encoding}

Type graph is an edge labeled \emph{and/or} graph. Types correspond to
vertices: product types are presented by \emph{and} vertices, union
nodes by \emph{or} vertices, and the unit type is an \emph{and} vertex
without children. 

We assume that we are given the type graph induced by the program.
A construction of the minimal type graph is relatively easy.  Given an
alphabet $\Sigma$, we propose the natural encoding of values of a
given type by words over $\Sigma$. For a given type, the set of all
its values is a prefix code.

If $t$ is a type, by $\texttt{id}(t)$ and $\texttt{skip}(t)$ we denote
the identity transducer (or CSM) on all values of $t$, and the
automaton accepting all values of $t$, respectively.  If $f$ is a
field name of type $t$, then $\texttt{proj}(t,f)$ denotes the transducer
taking any value of type $t$ and returning, if defined, the value of
the field $f$.  These transducers (CSMs) can be easily constructed
from the type graph.



\subsection{Denotational semantics of subexpressions by annotations}


With every subexpression $p$ we associate a pair $\sem{p}=(\pi,T)$, called
annotation of $p$, where $\pi$ is a list of free
references (with types) in $p$, and $T$ is a transducer (or CSM). If
$p$ is a closed subexpression, then $\pi$ is empty word, and $T$ is a
transducer implementing $p$. If $p$ contains some free occurences of
references, then $\pi$ represents thier order, possibly with
repetitions, which they should be supplied to the transducer $T$ in
order to be instantiated. Notice that an open subexpression can have
many different annotations, however a closed one (if the transducer is
normalized) will only have one.

The main difficulty in constracting a transducer semantics of a Kernel
subexpression is the treatement of free references. In order to
overcome the problem we give ourselves \emph{replicator} transducers.
We will write $\texttt{repl}(\pi_1,\pi_2)$, where $\pi_1=t_1t_2\ldots
t_n$ is a list of types, and $\pi_2=i_1i_2\ldots i_k$ is the list of
indexes, $i_j\in[1,n]$ for $j\in[1,k]$, to denote the transducer
accepting any input of type $\pi_1$ and transforming it into a (sub-)
list of the same values potentially with duplications and reorderings,
which is defined by $\pi_2$. For example, $\texttt{repl}((t),(1))$
corresponds to $\texttt{id}(t)$, and $\texttt{repl}((t),(1,1))$
corresponds to the transducer accepting any value $v$ of type $t$ and
producing $vv$.


\begin{description}

\item[identifier:] $\sem{\$x} \df{=} (\$x,\texttt{id}(t))$, where $t$
  is the type of $\$x$.

\item[variant:] $\sem{f~q} \df{=} (\pi,\texttt{code}(f:t)\cdot T)$, where
  $\sem{q}=(\pi,T)$, $t$ is the image type of $f~q$, and
  $\texttt{code}(f:t)$ is the constant transducer producing the
  encoding for field $f$ of type $t$.

\item[structure:] $\sem{\{f_1~q_1, f_2~q_2\}}\df{=}(\pi_1\pi_2,T_1\cdot T_2)$,
  where $\sem{q_1}=(\pi_1,T_1)$ and $\sem{q_2}=(\pi_2,T_2)$.

  The evaluation order of elements of a structure is given by the
  order of occurrences (from left to right) of sub-expressions. Thus,
  in general, we should write:
  \[
  \sem{\{f_{i_1}~q_{i_1},\ldots,f_{i_n}~q_{i_n}\}}
  \df{=}(\pi_{i_1}\cdots\pi_{i_n},(T_{i_1}\cdots T_{i_n})
  ::\texttt{repl}((t_{i_1} \ldots t_{i_n}),(1,\ldots,n))),
  \] 
  % 
  where $i_1,\ldots, i_n$ is a permutation of the encoding order
  $1,2,\ldots,n$ of the product type $\{t_1,\ldots,t_n\}$.


\item[projection:] $\sem{q.f}\df{=}(\pi,T :: \texttt{proj}(t,f))$,
  where $t$ is the image type of $q$ and $\sem{q}=(\pi,T)$.

\item[union:] $\sem{[q_1,q_2]}\df{=}(\pi_1\pi_2,T_1\cup T_2$, where
  $\sem{q_1}=(\pi_1,T_1)$, $\sem{q_2}=(\pi_2,T_2)$, and $T_1\cup T_2$
  denotes the priority merge of classifiers.

\item[composition:] $\sem{q_1 :: q_2}\df{=}(\pi_2\pi_1,
  (\texttt{id}(\pi_2)\cdot T_1) :: T_2)$, where
  $\sem{q_1}=(\pi_1,T_1)$ and $\sem{q_2}=(\pi_2,T_2)$. 

\item[abstraction:] $\sem{(\$x\,P \mapsto q)}\df{=}
  (\pi',\texttt{repl}(\pi'\, x, \omega)
  ::(\texttt{id}(\pi)\cdot\texttt{pattern}(P)) ::T)$, where
  $\sem{q}=(\pi,T)$, pattern $P$ is of type $t$, and $\pi'$ and
  $\omega$ are such that $x$ is not in $\pi'$ and
  $\texttt{repl}(\pi'\, x, \omega)$ corresponds to $\pi$, e.g., if
  $\pi=yxxyz$ then $\pi'=yz$ and $\omega=(1,3,3,1,2)$ are possible
  values.

\end{description}


\section{Transducer operations with respect to side effects}

\subsection{Product}

We have two cases for synchronization:
\begin{itemize}
\item concatenation $T_1;T_2$ --- every synchronization event of $T_1$
  precedes every event of $T_2$


\item intersection $T_1\cdot T_2$ --- an event in $T_1$ can occur only
  if the same (unified?) event occurs in $T_2$, or $T_2$ already
  finished (and vice-versa).
\end{itemize}

\subsection{Union}

\subsection{Pipeline}

As for concatenation product, in $T_1:: T_2$ every synchronization
event of $T_1$ precedes every event of $T_2$.

\end{document}



\end{document}

