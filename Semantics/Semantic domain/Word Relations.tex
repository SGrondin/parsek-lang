\documentclass{llncs}
\usepackage{graphicx}
\usepackage{color}
%\usepackage{shuffle}
\usepackage{amssymb}


%\theoremstyle{plain}
\newtheorem{thm}{Theorem}
\newtheorem{prop}[thm]{Proposition}
\newtheorem{cor}[thm]{Corollary}
\newtheorem{lem}[thm]{Lemma}
%\theoremstyle{definition}
\newtheorem{defn}{Definition}
%\theoremstyle{remark}
\newtheorem{conj}{Conjecture}
%\newtheorem{note}{Note}
%\theoremstyle{example}
\newtheorem{ex}{Example}

\newcommand{\udarrow}{{\uparrow\downarrow}}
\newcommand{\out}[1]{}
%\newcommand{\qed}{~\hfill{Q.E.D.}\\}
\newcommand{\df}[1]{\ \stackrel{\mathtt{def}}{#1}\ }
\newcommand{\fleche}[1]{\displaystyle \mathop {\mbox{\rightarrowfill}}^{#1} }

\title{Relations on words and FIFO channel systems}
\author{Wojciech Fraczak}
\institute{Universit\'e du Qu\'ebec en Outaouais, 
  Gatineau Qu\'ebec J8X 3X7, Canada
  \email{fraczak@uqo.ca}
}
\begin{document}
\maketitle
\begin{abstract}
  We reexamine the FIFO channel systems (also called Communicating
  Automata, Message-Passing Automata, or Data Flow Sequential
  Processes) as a virtual machine defining relations on
  words. Originally, the model consisted of a network of finite state
  machines communicating using reliable FIFO channels. The semantics
  of such a system was defined in terms of local state changes driven
  by the content of the FIFO channels. The model is simple,
  expressive, and apt to efficient distributed realizations.

  In this paper we investigate a compositional semantics for FIFO
  channel systems in terms of word relations. 
\end{abstract}

\section{Introduction}

The theory of formal languages and automata goes beyond languages seen
as subsets of finitely generated free monoids,
\cite{sakarovitch-03}. However, once the underlying monoid stops being
free, the notion of a corresponding automaton and its semantics begins
to be more complicated and less intuitive. In this paper we
characterize a class of word relations in order to use it as a
semantic domain for \emph{FIFO channel systems}, \cite{bz-83}. We
redefine the semantics of FIFO channel systems in terms of those
relations making it fully compositional. Our semantics, however, is
discriminative in the sense that it assumes that FIFO channel systems
have a special property, namely, they are deadlock free.  Intuitively,
a FIFO channel system is deadlock free, if it never stops in the
middle of producing its output. For example,
Kahn's deterministic systems, as defined in \cite{kahn-74}, are a spacial
case of deadlock free systems.

\section{Words, tuples, languages, and relations}

Let $S$ be a finite set. By $S^\ast$ we denote the set of all finite
sequences of elements of $S$, called \emph{words}. The empty word is
denoted by $\varepsilon$. Let $u$ and $w$ be words. We write $u\cdot
w$ (or $uw$) to denote the word concatenation and $|w|$ to denote the
length of $w$. 

By $(S^\ast)^n$, for a natural number $n$, we denote the set of all
$n$-tuples (vectors of size $n$) of words over $S$. If
$x=(w_1,\ldots,w_n)\in (S^\ast)^n$ and $y=(u_1,\ldots,u_m)\in
(S^\ast)^m$, then by $x\times y$ (or $x;y$) we mean
$(w_1,\ldots,w_n,u_1,\ldots,u_m)\in (S^\ast)^{n+m}$. If
$x=(w_1,\ldots,w_n)$ and $y=(u_1,\ldots,u_n)$ are both in
$(S^\ast)^n$, then $x\cdot y$ (or $xy$) denotes
$(w_1u_1,\ldots,w_nu_n)$. 

A set of words, i.e., a subset of $S^\ast$, is called a
\emph{language}. A set of tuples, i.e., a subset of $(S^\ast)^n$, is
called a \emph{relation}. Notice that a word is a special case of
tuple and a language is a special case of relation. Operations of
tuples extend to relations: $R_1\cdot R_2 = \{ xy \mid x\in R_1,
y\in R_2\}$, $R_1\times R_2 =  \{ x\times y \mid x\in R_1,
y\in R_2\}$.

We say that a tuple $u$ is a left factor of a tuple $w$, for $u,w\in
(S^\ast)^n$, if there exists $x\in(S^\ast)^n$ such that $ux=w$. We
will write $u\leq w$ to denote that $u$ is a left factor (also called
\emph{prefix}) of $w$ and $u\geq w$ to denote that $w$ is a prefix of $u$.

Let $R\subseteq (S^\ast)^n$ and $i\in\{1,\ldots,n\}$. By
${\downarrow_i}R$ (respectively, ${\uparrow_i}R$) we denote the prefix
(suffix\footnote{The name may be misleading in the sense that the
  ``suffix closure'' has nothing to do with the suffix, i.e., right
  factor. By the suffix closure of a language $L\subseteq S^\ast$ we
  mean $LS^\ast$ and by its prefix closure $L (S^\ast)^{-1}$.})
closure of $R$ on dimension $i$. I.e.:
\begin{eqnarray*}
  {\downarrow_i}R & \df{=} & 
  \left\{ (u\times x\times v) \mid u\in  (S^\ast)^{i-1}, x\in S^\ast, 
    \exists y \geq x ~ (u\times y\times v) \in R\right\} 
\out{ \\
  &  & \mbox{or, equivalently: }
  {\downarrow_i}R\df{=}R\cdot
  \left(\{\varepsilon^{i-1}\}\times S^\ast \times \{\varepsilon^{n-i}\}\right)^{-1}
}
\\
  {\uparrow_i}R & \df{=} & 
  \left\{ (u\times x\times v) \mid u\in  (S^\ast)^{i-1}, x\in S^\ast, 
    \exists y \leq x ~ (u\times y\times v) \in R\right\}
\out{
  \\
  & &\mbox{or, equivalently: }{\uparrow_i}R \df{=}
  R\cdot\left(\{\varepsilon^{i-1}\}\times S^\ast \times \{\varepsilon^{n-i}\}\right) 
}
\end{eqnarray*}

Notice that ${\downarrow_i\downarrow_j}R =
{\downarrow_j\downarrow_i}R$, ${\uparrow_i\uparrow_j}R =
{\uparrow_j\uparrow_i}R$, for all $i,j\in\{1,\ldots,n\}$, and
${\downarrow_i\uparrow_j}R = {\uparrow_j\downarrow_i}R$, for all
$i,j\in\{1,\ldots,n\}$ such that $i\neq j$. Thus, the notation
${\uparrow_{I}\downarrow_{J}}R$, where $I,J\subseteq\{1,\ldots,n\}$
and $I\cap J=\emptyset$ will be used to denote
$({\uparrow_{i_1}\uparrow_{i_2}\ldots\downarrow_{j_1}\downarrow_{j_2}\ldots}R)$
with $I=\{i_1,i_2,\ldots\}$ and $J=\{j_1,j_2,\ldots\}$.

Let $\mu:(\{1,\ldots,n\}\times S)^{\ast}\rightarrow (S^\ast)^{n}$ be
the usual monoid homomorphism which maps every
$(k,s)\in(\{1,\ldots,n\}\times S)$ into the tuple $(w_1,\ldots,w_n)$
of words over $S$ with $w_k=s$ and $w_i=\varepsilon$, for
$i\in\{1,\ldots,n\}\setminus\{k\}$.

Let $R\subseteq (S^{\ast})^m$ and $i\in\{1,\ldots,m\}$. We define the
hiding of dimension $i$ in $R$, denoted by $\mathit{HIDE}(R,i)$, as:
  \[
  \mathit{HIDE}(R,i)\df{=}\{u\times v \mid
  u\in(S^{\ast})^{i-1}, v\in(S^{\ast})^{m-i}, \exists x\in S^\ast,
  u\times x \times v \in L\}
\]

By $\mathit{HIDE}(R,I)$ for $I\subseteq\{1,\ldots,m\}$ we denote
simultaneous hiding of all dimensions from $I$.



\section{FIFO components}


Traditionally, a \emph{FIFO channel system} is a collection of
components (usually finite state machines) communicating with each other
through order-preserving unbounded point-to-point channels called FIFO
channels (for First-In-First-Out). In our modular presentation of FIFO
channel systems, a component is characterized by: (1) an
\emph{interface}, i.e., two disjoint sets of \emph{input} and
\emph{output ports}, which FIFO channels can connect to; and (2) a
\emph{word relation}, i.e., a set of tuples of words over
\emph{signals} read and written on the ports.

From an architectural (static) point of view, a FIFO channel consists
of components whose ports are connected by channels. In
Figure~\ref{fig:fifo-system}, the system $C$ consists of three
components $C_1$, $C_2$, and $C_3$. For example, component $C_1$ has
one input port (numbered by $1$) and two output ports (numbered by $2$
and $3$). In the figure, input ports are represented by black
squares and output ports by empty squares. The unconnected ports of
the components become the interface of the system. Such a system, in
turn, can be used as a component in a definition of another system.
\begin{figure}[htb]
  \centering
  \input{fifo-system.pstex_t}
  \caption{A FIFO channel system architecture.}
  \label{fig:fifo-system}
\end{figure}


In order to simplify the notation we assume that the set of
\emph{signals}, $S$, which can occur on a port is the same for all
ports. The semantics of a component $C$ with $m$ input ports and $n$
output ports, denoted $C(m,n)$, is defined as a relation $C \subseteq
(S^\ast)^{m+n}$. Intuitively, a tuple
$(w_1,\ldots,w_m,u_1,\ldots,u_n)$ in $C$, means that by writing
sequences of signals $w_1,\ldots,w_m$ on input ports $1,\ldots,m$, we
may be able to read sequences of signals $u_1,\ldots,u_n$ on output
ports $m+1,\ldots,m+n$, respectively.

We will write $(x;y)\in C$ in order to explicitly separate the tuple
$x\times y \in C$ into the ``writing tuple'' $x\in (S^\ast)^{m}$ and
the ``reading tuple'' $y\in (S^\ast)^{n}$, as we assume that the input
ports are numbered from $1$ to $m$ and output ports from $m+1$ to
$m+n$.

\begin{defn} 
  A FIFO component $C$ with $m\geq 0$ input ports and $n\geq 0$ output
  ports is a word relation $C\subseteq (S^\ast)^{m+n}$ such that:
  \[ 
  {\uparrow_{\{1,\ldots,m\}}\downarrow_{\{m+1,\ldots,m+n\}}}(C
  \cup\{ \varepsilon^{m+n}\}) ~ = ~ C
  \]
\end{defn}

In other words, a relation is a \emph{FIFO component} if it is
defined for the empty input, it is suffix-closed on input, and it is
prefix-closed on output.

The above constraints capture the restrictions for modular
implementations imposed by the real asynchronous world in the spirit
of the Kahn's deterministic processes, \cite{kahn-74}, and the
Josephs's receptive and data flow processes,
\cite{josephs-92,josephs-04}. The environment, i.e., a user of a
component, can always send signals to the component (the component
doesn't control what the user is pushing into its input ports) and it
reads signals on the output ports of the component one-by-one, e.g., a
possibility of reading a word $w$ on a port, implies possibilities of
reading any prefix of $w$ on the port.

In a context of known number $m$ of input ports and $n$ of output
ports, we will write ${\udarrow}R$ rather than
${\uparrow_{\{1,\ldots,m\}}\downarrow_{\{m+1,\ldots,m+n\}}}(R \cup\{
\varepsilon^{m+n}\})$. Intuitively, ${\udarrow}R$ denotes
the FIFO component closure of any relation $R\in (S^\ast)^{m+n}$.

\begin{prop}\label{prop:lattice}
  FIFO components are closed under the union and the intersection, i.e.,
  if $C_1$ and $C_2$ are FIFO components with $m$ input and
  $n$ output ports, then so are $C_1\cap C_2$ and $C_1\cup C_2$. 
\end{prop}

\begin{proof}
  Directly from the definition of the FIFO component. In fact, FIFO
  components seen as sets constitute a complete lattice. \qed
\end{proof}

\begin{defn}
  Let $C\subseteq (S^\ast)^{m+n}$ be a FIFO component. We say that a
  word $t\in (\{1,\ldots,m+n\},S)^\ast$ is a \emph{trace} of $C$ if
  for every prefix $t'$ of $t$, $\mu(t')\in C$.  By $\mathcal{T}(C)$
  we will denote the set of all traces of $C$.
\end{defn}

\begin{prop}
  Let $C$ be a FIFO component.  For every $x\in C$ there exists a
  trace $t\in\mathcal{T}(C)$, such that $\mu(t)=x$.
\end{prop}

\begin{proof}
  Assume that $C$ has $m$ input and $n$ output ports, and $x=(u;v)\in
  C$, where $u\in(S^\ast)^m$ is the $m$-tuple of input words and
  $v\in(S^\ast)^m$ is the $n$-tuple of output words in $x$.  The sets
  $T_1=\mu^{-1}(u\times \varepsilon^n)$ and
  $T_2=\mu^{-1}(\varepsilon^m\times v)$ are both non-empty. Since
  $C={\udarrow}C$, for every $t\in T_1\cdot T_2$ and every
  prefix $t'\leq t$, $\mu(t')\in C$.  \qed
\end{proof}

Intuitively, a trace represents a possible sequence of reads and
writes of a sequential ``user'' of the component. 

The usual \emph{reordering} relation on traces $\ltimes$ (e.g.,
\cite{josephs-04}) can be defined as the reflexive and transitive closure of:
\[ 
\{(u (p,s) (q,t) v, u (q,t) (p,s) v) \mid p\neq q \mbox{ and } (p\leq
m \mbox{ or } q > m) \}
\] 
where $P=\{1,\ldots,m+n\}$, $u,v\in (P\times S)^\ast$, and
$(p,s),(q,t)\in P\times S$.

We introduce the \emph{feasibility} relation, a pre-order relation
over $\{1,\ldots,m+n\}\times S$, in order to captures the implication
on ``possibility'' of occurring between traces. I.e., we will write
$u\preceq v$, if $u\in {\udarrow}t$, where ${\udarrow}t \df{=}
\mathcal{T}({\udarrow}\{\mu(t)\})$. Intuitively, $v \preceq u$ states
that every system admitting a trace $u$ will also admit trace $v$. For
example, $v\leq u$ or $v \ltimes u$ implies $v\preceq u$.

\begin{prop}\label{prop:traces}
  Let $T\subseteq (\{1,\ldots,m+n\}\times S)^\ast$ be a set of traces
  of a FIFO component with $m$ input and $n$ output ports. $T$ is
  closed with respect to the \emph{feasibility} relation, i.e., if
  $t\in T$ then $t'\in T$, for every $t'\preceq t$.
\end{prop}

A pair $(p,s)\in \{1,\ldots,m+n\}\times S$ will be written as $p?s$ if
$p\leq m$ and $p!s$ if $p> m$, in order to make it explicit that $p$ is
an input or output port, respectively.

By ${\udarrow}T$ we denote the closure of set of traces $T$ with respect to
$\preceq$, i.e., ${\udarrow}T\df{=}\{{\udarrow}t\mid t\in T\}$.  In
\cite{josephs-04}, a non-empty set of traces $T$ closed with respect
to the \emph{feasibility} relation, i.e., $T={\udarrow}T$, is called
``\emph{healthy}''.

For every FIFO component $C$, $\mathcal{T}(C)$ is a healthy set of
traces. However, it is important to notice that not every healthy set
of traces corresponds to the set of traces of a FIFO component, since
in general we have $\mathcal{T}(C_1)\cup\mathcal{T}(C_2) \subsetneq
\mathcal{T}(C_1\cup C_2)$.  For example, following the Brock-Ackerman
anomaly from \cite{ba-81}, consider $C_1=
{\udarrow}\{(\varepsilon,a)\}$ and $C_2 ={\udarrow}\{(b,aa)\}$, i.e.,
two minimal FIFO components containing trace $t_1=(2!a)$ and
$t_2=(1?b)(2!a)(2!a)$, respectively.  FIFO component $C=C_1\cup C_2$
includes the trace $t=(2!a)(1?b)(2!a)$, since
$(b,aa),(b,a),(\varepsilon,a)$ are all in $C$.  However, $t\notin
\mathcal{T}(C_1)\cup \mathcal{T}(C_2)$.

\out{
By $\max{T}$ we denote all maximum elements of $T$ with respect to
$\preceq$, i.e.,
\[ 
\max{T} \df{=} \{t \in T \mid \forall t'\in T ~ t\preceq t'
\Rightarrow t\simeq t'\}.
\]
  
The equivalence relation over traces generated by $\preceq$ will be
denoted by $\simeq$.
  
Within an equivalence class $[|t|]\df{=}\{x\in (\{1,\ldots,m+n\}\times
S)^\ast\mid x\simeq t\}$ the natural candidate as a
representative\footnote{The choice of the representative is arbitrary
  as long as we take it from the finite set of the shortest elements
  of $[|t|]$.} of $[|t|]$, denoted by $\mathrm{rep}([|t|])$, is a
trace which is of the shortest length and which, for example, reads
and writes in order of port numbers. E.g., for $t=2?a\cdot 1?a\cdot
4!a\cdot 3!b\cdot 1?b$, we would have $\mathrm{rep}([|t|]) = 1?a\cdot
2?a\cdot 3!b\cdot 4!a$.
  
We denote $\mathrm{rep}(T)$ the set of representatives of the
equivalence classes generated by elements of $T$, i.e.,
$\mathrm{rep}(T) \df{=} \{\mathrm{rep}([|t|]) \mid t\in\max{T}\}$.
Notice that in general ${\udarrow}\mathrm{rep}(T)\subsetneq
{\udarrow}T$. For example, ${\udarrow}\mathrm{rep}(T)\neq
{\udarrow}T$, for $T=(3!a)^{\ast}$, since $\max{T}=\emptyset$.
}

\subsection{Automata representation of FIFO components}
\label{sec:fifo-state-machine}

An automaton, $\mathcal{A}=(Q,\Sigma,I,F,E)$, consists of a set $Q$ of
\emph{states}, a finite alphabet $\Sigma$, a set of initial states
$I\subseteq Q$, a set of final states $F\subseteq Q$, and a transition
relation $E\subseteq Q\times(\Sigma\cup\{\varepsilon\})\times Q$. An
automaton is finite if its set of states $Q$ is finite.

Automaton $\mathcal{A}=(Q,\Sigma,I,F,E)$ defines a language
$L(\mathcal{A})$ over $\Sigma$. A word $w\in\Sigma^{\ast}$ is in
$L(\mathcal{A})$ if there is a path in $\mathcal{A}$ starting in a
initial state $q_I\in I$ and ending in a final state $q_F\in F$
labeled by $w$, i.e., there exists a finite (possibly empty) sequence
of transitions $(q_I,e_0,q_1), (q_1,e_1,q_2),\ldots, (q_k,e_k,q_F)$,
such that $w=e_0e_1\ldots e_k$. 

In our case, the alphabet $\Sigma$ consists of port-signal pairs,
i.e., $\Sigma=\{1,\ldots,m\}\times S$, where $m$ is a natural number
(of ports) and $S$ is a finite set of signals. Therefore, an automaton
$\mathcal{A}=(Q,\{1,\ldots,m\}\times S,I,F,E)$ defines also a
relation $R(\mathcal{A})\df{=}\mu(L(\mathcal{A}))\subseteq (S^\ast)^m$.

A relation $R\subseteq (S^\ast)^m$ is called \emph{rational} if there
is a finite automaton $\mathcal{A}=(Q,\{1,\ldots,m\}\times S,I,F,E)$
such that $R=R(\mathcal{A})$. 


\begin{lem}\label{lem:updown-rat}
  If $R\subseteq(S^\ast)^{n}$ is rational then ${\uparrow_i}R$ and
  ${\downarrow_i}R$, for $i\in \{1,\ldots,n\}$, are rational too.
\end{lem}

\begin{proof}
  $R$ is rational so there exists a finite automaton
  $A=(Q,\{1,\ldots,m\}\times S, I, F, E)$ for $R$. An automaton for
    ${\downarrow_i}R$ is $(Q\cup Q',\{1,\ldots,m\}\times S, I\cup I',
      F\cup F', E\cup E'\cup E'')$, where
      \begin{itemize}
      \item $Q'=c(Q)$ is a disjoint copy of $Q$ by a bijection
        $c:Q\rightarrow Q'$;
      \item $I'=c(I)$ and $F'= c(F)$;
      \item $E'=\{(q,\varepsilon,c(q))\mid q\in Q\}\cup
        \{(c(q),\varepsilon,c(q'))\mid \exists (q,(i,s),q')\in E\}$;
      \item $E''=\{(c(q),(j,s),c(q'))\mid j\neq i, (q,(j,s),q')\in E\}$.
      \end{itemize}

      Let $I=\{\varepsilon^{i-1}\}\times S^\ast \times
      \{\varepsilon^{n-i}\}$. Since $R$ and $I$ are rational, so is
      ${\uparrow_i}R = R\cdot I$. \qed
\end{proof}


Directly form Lemma~\ref{lem:updown-rat} we have:
\begin{thm}
  \label{thm:rat}
  If $R\subseteq(S^\ast)^{m+n}$ is rational then the FIFO component
  closure of $R$, i.e., ${\udarrow}R$, is rational too.
\end{thm}
Thanks to Theorem~\ref{thm:rat}, any rational $R\subseteq (S^\ast)^{m+n}$
defines a rational FIFO component ${\udarrow}R$.


\out{ An example of a finite state FIFO component is the ``copy''
  component: $\mathtt{copy}(1,1)\df{=} {\udarrow}\{(u,u) \mid u \in
  S^\ast\}$, with one input and one output port.
% 
}

An automaton $A=(Q,\Sigma,I,F,E)$ will be called a \emph{FIFO state
  machine} if $\Sigma=\{1,\ldots,m+n\}\times S$, $F=Q$, and the
automaton is \emph{deterministic}, i.e.:
\begin{itemize}
\item $I=\{q_0\}$, i.e., there is one initial state;
\item $E\subseteq Q\times \Sigma\times Q$, i.e., there is no
  \emph{silent} transitions;
\item for every $(q_1,(p_1,s_1),q'_1), (q_2,(p_2,s_2),q'_2)\in E$, if
  $q_1=q_2$ then $p_1=p_2$, i.e., in every state we read or write only
  on one port (one dimension);
\item for every $(q_1,(p_1,s_1),q'_1), (q_2,(p_2,s_2),q'_2)\in E$, if
  $q_1=q_2$ and $s_1=s_2$ then $q'_1=q'_2$, i.e., in every state what we 
  read or write defines the unique next state.
\end{itemize} 

An automaton which is a FIFO state machine will be represented by a
quadruple $A=(q_o,\Sigma,I,E)$.

A FIFO state machine can be seen as an abstract implementation of a
component consisting of a control state machine together with FIFO
queues, one for every port. The global state, called
\emph{configuration}, of the system implementing the component is
described by the local state of the control state machine together
with the content of the queues. The environment (user of the
component) communicates with the component by writing signals into
input port queues and reading signals from the output port queues.

The finite FIFO state machines are of interest for us
because of two reasons. Firstly, they are effectively\footnote{By
  ``effectively implementable'' we mean that a deadlock-free program
  or a device for the FIFO component can be derived directly from its
  definition.}  implementable in software or in hardware. Secondly,
the problem of equivalence of two finite FIFO state
machines is decidable\footnote{The statement follows (not so directly)
  from the decidability result for the equivalence of deterministic
  multi-tape finite automata, \cite{hk-91}.}.


The behavior of a FIFO state machine can be described in
terms of a \emph{labeled transition system} (\emph{lts}). An
\emph{lts} is a triple $(G,T,\lambda)$, where $G$ is the set of global
states (configurations), $\lambda$ is a finite set of \emph{labels},
and $T\subseteq G\times\lambda\times G$, is the set of (global)
transitions. A global transition $t=(g_1,l,g_2)\in T$ will be denoted
by $g_1\fleche{l}g_2$, and its label $l$ by $\lambda(t)$.

For the \emph{lts} associated with a FIFO state machine
$\mathcal{F}(m,n)=(\{q_0\},\Sigma,I,E)$, the global states are $m+n+1$
tuples, $G\subseteq(S^\ast)^{m+n}\times Q$. The initial global state of the \emph{lts} is
$(\varepsilon,\ldots,\varepsilon,q_0)$; it represents
the initial configuration where all queues are empty, and the local
control state of the component is in the initial state. The global
transitions of the \emph{lts}, $T\subseteq G\times\lambda\times G$,
represent configuration changes made by the environment (when a new
signal is appended into an input port queue, or a signal is read from
an output port queue) or by the control state machine (when a local
transition $e\in E$ is executed). Thus, $\lambda$ labels every global
transition $t\in T$ by $\{1,\ldots,m+n\}\times S$ or by $E$, i.e.,
$\lambda=(\{1,\ldots,m+n\}\times S) \cup E$.  The \emph{lts}
$(G,T,\lambda)$ for a FIFO state machine
$\mathcal{F}(m,n)=(Q,\Sigma,I,E)$ is defined by the following rules
(with $i\in\{1,\ldots,m+n\}$, $\alpha_i\in S^\ast$, $s\in S$, $q,q'\in Q$):
\begin{itemize}
\item
  $(\alpha_1,\ldots,\alpha_j,\ldots,\alpha_{m+n}, q)
  \fleche{~j?s~}
  (\alpha_1,\ldots,\alpha_j\cdot s,\ldots,\alpha_{m+n}, q)$

  for every $j\in\{1,\ldots,m\}$;

\item
  $(\alpha_1,\ldots,s\cdot\alpha_j,\ldots,\alpha_{m+n},q)
  \fleche{~j!s~}
  (\alpha_1,\ldots,\alpha_j,\ldots,\alpha_{m+n}, q)$

  for every $j\in\{m+1,\ldots,m+n\}$;

\item $(\alpha_1,\ldots,s\cdot\alpha_j,\ldots,\alpha_{m+n},q)
  \fleche{~e~}
  (\alpha_1,\ldots,\alpha_j,\ldots,\alpha_{m+n},q')$,

  for every $j\in\{1,\ldots,m\}$ and $e\in E$ such that $e=(q,j?s,q')$;

\item $(\alpha_1,\ldots,\alpha_j,\ldots,\alpha_{m+n},q)\fleche{~e~}
  (\alpha_1,\ldots,\alpha_j\cdot s,\ldots,\alpha_{m+n},q')$,

  for every $j\in\{m+1,\ldots,m+n\}$ and $e\in E$ such that $e=(q,j!s,q')$.
\end{itemize}

By $\Pi(\mathcal{F})$ we will denote the set of all paths in the above
\emph{lts} starting in the initial configuration. By
$\lambda(\pi)\subseteq(E\cup\{1,\ldots,m+n\}\times S)^\ast$, for
$\pi\in\Pi$, we denote the label of $\pi$, i.e., the concatenations of
all labels of the global transitions of $\pi$. Also, by
$\lambda_i(\pi)\subseteq E^\ast$ and by $\lambda_e(\pi)\subseteq
(\{1,\ldots,m+n\}\times S)^\ast$ we denote the ``\emph{internal}'' and
``\emph{external}'' label of $\pi$, i.e., where all occurrences of
$\{1,\ldots,m+n\}\times S$ or of $E$, respectively, are erased from
$\lambda(\pi)$.

\begin{thm}
  Let $\mathcal{F}$ be a FIFO state machine defining FIFO component
  $C$. We have
  \[ \mathcal{T}(C) = \lambda_e(\Pi(\mathcal{F})) 
  \]
  where $\lambda_e(\Pi(\mathcal{F})) \df{=}
  \{\lambda_e(\pi)\mid\pi\in\Pi(\mathcal{F})\}$.
\end{thm}

\begin{proof}
  We need to prove that:
  \begin{itemize}
  \item for every $t\in \mathcal{T}(C)$ there exists
    $\pi\in\Pi(\mathcal{F})$ such that $t = \lambda_e(\pi)$; and
  \item for every $\pi\in\Pi(\mathcal{F})$ we have $\lambda_e(\pi)\in
    \mathcal{T}(C)$.
  \end{itemize}

  ...
\end{proof}

A FIFO state machine $\mathcal{F}(m,n)=(Q,\Sigma,I,E)$ is
\emph{deterministic} if:
\begin{itemize}
\item $I=\{q_0\}$, i.e., there is one initial state;
\item $E\subseteq Q\times \Sigma\times Q$, i.e., there is no
  \emph{silent} transitions;
\item for every $(q_1,(p_1,s_1),q'_1), (q_2,(p_2,s_2),q'_2)\in E$, if
  $q_1=q_2$ then $p_1=p_2$, i.e., in every state we read or write only
  on one port (one dimension);
\item for every $(q_1,(p_1,s_1),q'_1), (q_2,(p_2,s_2),q'_2)\in E$, if
  $q_1=q_2$ and $s_1=s_2$ then $q'_1=q'_2$, i.e., in every state what we 
  read or write defines the unique next state.
\end{itemize} 

The finite deterministic FIFO state machines are of interest for us
because of two reasons. Firstly, they are effectively\footnote{By
  ``effectively implementable'' we mean that a deadlock-free program
  or a device for the FIFO component can be derived directly from its
  definition.}  implementable in software or in hardware. Secondly,
the problem of equivalence of two finite deterministic FIFO state
machines is decidable\footnote{The statement follows (not so directly)
  from the decidability result for the equivalence of deterministic
  multi-tape finite automata, \cite{hk-91}.}.

\section{Gathering and wiring FIFO components}

A FIFO channel system is a network of FIFO components. Any such finite
network can be described as the result of \emph{grouping} and
\emph{wiring} components. The grouping operation is defined as a
binary operation corresponding to the \emph{parallel composition}.
The semantics of two FIFO components, $C_1$ and $C_2$, grouped
together is a new component $(C_1\parallel C_2)\subseteq
(S^\ast)^{m_1+m_2+n_1+n_2}$ with $m_1+m_2$ input ports and $n_1+n_2$
output ports, where $m_i, n_i$ are the numbers of input and output ports of
$C_i$, for $i\in\{1,2\}$.
\[(C_1\parallel C_2) \df{=} \{(x_1\times x_2;y_1\times y_2) \mid
(x_1;y_1)\in C_1, (x_2;y_2)\in C_2\}.
\]
The semantics of $(C_1\parallel C_2)$ corresponds to $C_1\times C_2$
followed by a reordering of the words in the tuples which moves the
input words of $C_2$ in front of the output words of $C_1$.  With an
``appropriate''\out{\footnote{The port renaming $ren(p)$ is $p$ if $p$ is an
  input port in $C_1$, $p+m_2$ if $p$ is an output port in $C_1$,
  $p+m_1$ if $p$ is an input port in $C_2$, and $p+m_1+n_1$ if $p$ is
  an output port in $C_2$, where $m_1$, $n_1$ are the numbers of input
  and output ports in $C_1$, respectively, and $m_2$ is the number of
  input ports in $C_2$.}}  port renaming of traces of $C_1$ and $C_2$,
the set of traces of $(C_1\parallel C_2)$ corresponds to the shuffle
of the renamed traces of $C_1$ with the renamed traces of $C_2$,

\begin{prop}\label{prop:par-assoc}
  The grouping operation is associative.
\end{prop}
\begin{prop}\label{prop:par-increasing}
  The grouping operation is increasing, i.e., $C_1\subseteq C_2$
  implies $(C_1\parallel C)\subseteq (C_2\parallel C)$ and
  $(C\parallel C_1)\subseteq (C\parallel C_2)$.
\end{prop}

\noindent
Wiring operation, $C\circ[p\leftarrow q]$, takes a FIFO
component $C$, one of its input ports $p$, and one of its output ports
$q$, and connects the ports together. The connected ports disappear
from the interface of $C\circ[p\leftarrow q]$. The semantics of
$C\circ[p\leftarrow q]$ is built out of those traces of $C$ in which
every read of a signal on port $p$ is preceded by a write of the
signal to the port $q$. The set of such traces can be defined as
follows: 
\[\mathtt{SYNCH}(C,p\leftarrow q) \df{=}
\{t\in \mathcal{T}(C) \mid
  \forall x\leq t, ~ x|_{p} \leq x|_{q} \}~,
\]
where $t|_{j}$, for $t\in(\{1, \ldots, n \}\times S)^\ast$ and
$j\in\{1,\ldots,n\}$, denotes the word over $S$ obtained from $t$ by
its projection on port $j$, i.e., $\varepsilon|_{j}=\varepsilon$,
$((i,s)\cdot t)|_{j}= s\cdot t|_{j}$ if $j= i$, and $((i,s)\cdot
t)|_{j}=t|_{j}$ if $j\neq i$. Intuitively, a trace $t$ of $C$ is in
$\mathtt{SYNCH}(C,p\leftarrow q)$ if in every prefix $x$ of $t$, the
word on port $p$ is a prefix of the word on port $q$.

Finally we can write:
\[
C\circ[p\leftarrow q] \df{=} 
\mathit{HIDE}\left(
  \mu\left(\mathtt{SYNCH}(C,p\leftarrow q) 
  \right), 
  \{p,q\}
\right).
\]

\begin{prop}\label{prop:wire-comm}
  The wiring operation is ``commutative'', i.e., if $p_1< p_2 <q_1<
  q_2$ then
  \[C\circ[p_1\leftarrow q_1] \circ [p_2-1\leftarrow q_2-2] =
  C\circ[p_2\leftarrow q_2] \circ [p_1\leftarrow q'_1-1],\] 
  and if $p_1< p_2 <q_2< q_1$ then
  \[C\circ[p_1\leftarrow q_1] \circ [p_2-1\leftarrow q_2-1] =
  C\circ[p_2\leftarrow q_2] \circ [p_1\leftarrow q'_1-2].\]
\end{prop}

The proposition can be proved using the following simple
observation.
\begin{lem}
  Let $C(m,n)$ be a component with at least two input ports $p_1$, $p_2$
  and two output ports $q_1$, $q_2$, i.e., let $1\leq p_1< p_2\leq m < q_1 <
  q2 \leq m+n$.

  The component $(C\circ[p_1\leftarrow q_1]) \circ
  [p_2-1\leftarrow q_2-2]$ is equal to:
%%%
  \[
  \mathit{HIDE}\left( \mu
    (\mathtt{SYNCH}(C,p_1\leftarrow q_1)
    \cap
    \mathtt{SYNCH}(C,p_2\leftarrow q_2)), 
    \{p_1,p_2,q_1,q_2\}
  \right),
\]
  and component $(C\circ[p_1\leftarrow q_2]) \circ
  [p_2-1\leftarrow q_2-1]$ is equal to:
  \[
  \mathit{HIDE}\left(\mu(
    \mathtt{SYNCH}(C,p_1\leftarrow   q_2)
    \cap
    \mathtt{SYNCH}(C,p_2\leftarrow q_1)),
    \{p_1,p_2,q_1,q_2\}
  \right)~.
\]
\end{lem}



The system from Figure~\ref{fig:fifo-system} could be described by the
following \emph{grouping} and \emph{wiring} expression:
$\left(\left(\left((C_1\parallel (C_2\parallel C_3)) \circ
      [2\leftarrow 7]\right) \circ [3\leftarrow 5]\right) \circ
  [2\leftarrow 6]\right) \circ [2\leftarrow 3]$.


Notice that the port numbering changes after every operation. Thanks
to Propositions~\ref{prop:par-assoc} and \ref{prop:wire-comm}, we can
introduce the simultaneous wiring operation in order to make the
expression closer to its graphical representation (see
Figure~\ref{fig:fifo-system-2}):
\[
(C_1\parallel C_2\parallel C_3) 
\circ [2\leftarrow 7, 4\leftarrow 6,
3\leftarrow 10, 5\leftarrow 9].
\]
%
\begin{figure}[htb]
  \centering
  \input{fifo-system-2.pstex_t}
  \caption{Gathering $(C_1\parallel C_2\parallel C_3)$ and wiring
    {\scriptsize $[2\leftarrow 7, 4\leftarrow 6, 3\leftarrow 10,
      5\leftarrow 9]$}, or equivalently {\scriptsize $[C_2.1\leftarrow
        C_1.3, C_3.1\leftarrow C_1.2, C_2.2\leftarrow C_3.4,
        C_3.2\leftarrow C_3.3]$}, yields component $C(1,1)$ from
    Figure~\ref{fig:fifo-system}.}
  \label{fig:fifo-system-2}
\end{figure}
%
We will also allow an even more direct notation, by allowing composed
port identifiers of the form \texttt{component.local-port-number} when
a gathering is immediately followed by a wiring. Thus, the system from
Figure~\ref{fig:fifo-system} could be written as:
$(C_1\parallel C_2\parallel C_3) 
\circ [\mbox{\footnotesize $C_2.1\leftarrow C_1.3, C_3.1\leftarrow C_1.2,
C_2.2\leftarrow C_3.4, C_3.2\leftarrow C_3.3$}]$.

\out{ That last notation allows us to write the following equation:
  \[((C_1 \parallel C_2)\circ [H_1,\ldots, H_k] ) \parallel C_3 =
  (C_1 \parallel C_2 \parallel C_3) \circ [H_1,\ldots, H_k],
  \]
  whereas in fact we should have written  
  \[((C_1 \parallel C_2)\circ [H_1,\ldots, H_k] ) \parallel C_3 =
  (C_1 \parallel C_2 \parallel C_3) \circ [\tilde{H_1},\ldots, \tilde{H_k}],
  \]
  since the port positions are changing after every grouping; by
  $\tilde{H}$ we mean the wiring $H$ with the appropriate renumbering of
  the corresponding ports.
}



\begin{prop}\label{prop:wire-increasing} 
  The wiring operation is increasing, i.e., for any wiring $H$,
  $C_1\subseteq C_2$ implies $C_1\circ H \subseteq C_2\circ H$.
\end{prop}
\begin{proof} 
  It is enough to check that $\mathtt{SYNC}$ is increasing. \qed
\end{proof}

\subsection{Recursive definitions of FIFO channel systems}

The compositional semantics allows us to introduce recursion in system
descriptions. Consider the system $X(1,1)$ from
Figure~\ref{fig:reverse} which, intuitively, reads one word over two
letter alphabet, $a$, $b$, terminated by an end-marker, $\$$, and
outputs the reverse of its input.

\begin{figure}[htb]
  \centering
  \input{reverse.pstex_t}
  \caption{Recursive description of $X= (X\parallel C)\circ
    [\mbox{\footnotesize $X?1\leftarrow C!3, C?2\leftarrow X!2$}]$,
    left, and the automaton defining component $C$, right.}
  \label{fig:reverse}
\end{figure}

In general, a recursive definition $X=\Psi(X)$, where $\Psi(\ldots)$
is a ``gathering-wiring'' expression, can be interpreted as the
smallest (or the biggest) by inclusion component satisfying the
equation. Since any such expression is \emph{increasing},
Propositions~\ref{prop:par-increasing} and \ref{prop:wire-increasing},
i.e., $C_1\subseteq C_2$ implies $\Psi(C_1)\subseteq \Psi(C_2)$, and
FIFO components constitute a complete lattice,
Proposition~\ref{prop:lattice}, by Knaster-Tarski theorem
\cite{tarski-55}, such FIFO components exist. For example, the
smallest solution can be described as follows:
\[ X\df{=} \bigcup_{i\geq 0} X_i ~,
\] 
where $X_0\df{=}\udarrow(\varepsilon^{m+n})$ and
$X_{k+1}\df{=}\Psi(X_k)$, for $k\geq 0$.


\subsection{Execution model}

In this section we assume that a FIFO channel system is built out of a
finite set of finite FIFO state machines in the spirit of
\cite{bz-83}. We recall the execution model of such a system by
defining an infinite FIFO state machine implementing it.  

\out{Then we show that the semantics of the system in terms of word
  relations coincides with this execution model.}

Let $F_i(m_i,n_i)=(Q_i,\Sigma_i,I_i,E_i)$, for $i\in\{1,\ldots,k\}$,
$\Sigma_i=\{1,\ldots,m_i+n_i\}\times S$, be a finite collection of
components given by their finite FIFO state machines, and
$H_i=\{(f_i,p_i),(f'_i,p'_i)\}$, for $i\in\{1,\ldots,l\}$,
$f_i,f'_i\in \{1,\ldots,k\}$,
$p_i\in\{m_{f_i}+1,\ldots,m_{f_i}+n_{f_i}\}$, $p'_i\in
\{1,\ldots,m_{f'_i}\}$, be a set of point-to-point FIFO
channels. Intuitively, the system $\Omega$ is defined as:
\[
 \Omega = (F_1\parallel \ldots \parallel F_k)\circ [H_1,\ldots, H_l].
\]
Components are numbered from $1$ to $k$ and channels from $1$ to
$l$. Component number $i$ is defined by FIFO state machine $F_i$ with
$m_i$ input and $n_i$ output ports. A channel number $j$, $H_j$, is defined by
two pairs $\{(f_j,p_j),(f'_j,p'_j)\}$ of ports; $H_j$ connects
port $p_j$ of component $f_j$ with port $p'_j$ of component $f'_j$. By
``point-to-point channels'' we mean that one port of a component is
connected to at most one channel, i.e., if $i\neq j$ then $H_i\cap
H_j=\emptyset$ for all $i,j\in\{1,\ldots,l\}$.

Let $A_{\Omega}= (Q,\Sigma,I,Q,E)$ be the following (infinite)
automaton, where
\begin{itemize}
\item The set of states $Q$ is
\[Q\df{=} Q_1\times Q_2\times\ldots\times Q_k\times (S^\ast)^l\]
\item The alphabet $\Sigma$ consists of local transitions, i.e.:
\[\Sigma\df{=}E_1\cup E_2\cup \ldots\cup E_k\]
\item The set of initial states $I$ is defined as:
  \[I = I_1\times I_2\times\ldots\times I_k\times (\{\varepsilon\})^l\]
\item The transition relation $E$ consists of \emph{steps}. Every step
  corresponds to a firing of a local transition which may change only
  one local state and one channel:
  \[
  ((q_1,\ldots,q_k,f_1,\ldots,f_l),e,(q'_1,\ldots,q'_k,f'_1,\ldots,f'_l))
  \in E
  \] 
  if and only if there exists a component $i\in\{1,\ldots,k\}$ and a
  transition $e=(q,\alpha,q')\in E_i$ such that:
  \begin{enumerate}
  \item $q_i=q$, $q'_i=q'$, and $q_j=q'_j$ for all
    $j\in\{1,\ldots,k\}\setminus\{i\}$;
  \item If $\alpha=\varepsilon$, i.e., $e$ is a silent transition,
    then $f'_j=f_j$, for all $j\in\{1,\ldots,l\}$;
  \item If $\alpha=p!s$, i.e., transition $e$ writes signal $s\in S$
    to output port $p\in\{m_i+1,\ldots,m_i+n_i\}$, then, for all
    $j\in\{1,\ldots,l\}$,
    \[
    f'_j =\left\{\begin{array}{ll}
        f_j\cdot s & \mbox{ if channel $j$ connects $p$, 
          i.e., $(i,p)\in H_j$ } \\
        f_j & \mbox{ otherwise}
      \end{array}
    \right.\]
  \item If $\alpha=p?s$, i.e., transition $e$ reads signal $s\in S$ from
    input port $p\in\{1,\ldots,m_i\}$, then, for all
    $j\in\{1,\ldots,l\}$,
    \[
    f_j =\left\{\begin{array}{ll}
        s \cdot f'_j & \mbox{ if channel $j$ connects $p$, 
          i.e., $(i,p)\in H_j$ } \\
        f'_j & \mbox{ otherwise}
      \end{array}
    \right.\]
  \end{enumerate}
\end{itemize}

Let $\phi:\Sigma\rightarrow (\{1,\ldots,n\}\times S)\cup
\{\varepsilon\}$ be the relabeling of transitions from local
transitions of the components into ports of $\Omega$, with $n$ being
the number of unconnected ports in the system. Intuitively, if
$e=(q,\alpha,q')$ is a local transition of component $i$ with
$\alpha=\varepsilon$ or $\alpha$ writing/reading on a connected port,
then $\phi(e)=\varepsilon$. Otherwise, $e$ is reading/writing
signal $s$ on an interface port $p$ of $\Omega$, and thus
$\phi(e)=(p,s)$.

Finally, from $A_{\Omega}=(Q,\Sigma,I,Q,E)$ we define the FIFO state
machine for $\Omega$ as $\mathcal{F}_{\Omega}=(Q,\{1,\ldots,n\}\times
S,I,\phi(E))$, where $\phi((q,e,q'))\df{=}(q,\phi(e),q')$.

\begin{thm}
  Let $C_{\Omega}$ be a FIFO component semantics of $\Omega$ and
  $\mathcal{F}_{\Omega}=(Q,\Sigma,I,E)$ be the FIFO state machine
  semantics for $\Omega$ as defined in this section.  We have:
  \[
  \lambda_e(\Pi(\mathcal{F}_{\Omega})) = \mathcal{T}(\Omega),
  \]
  where $\Pi(\mathcal{F}_{\Omega})$ is the set of paths of the
  \emph{lts} for $\mathcal{F}_{\Omega}$ as defined in
  Section~\ref{sec:fifo-state-machine}.
\end{thm}


The theorem (whose proof doesn't fit into the paper) says that
the FIFO channel system semantics in terms of word relation captures
the visible (external) part of the traditional execution model for
such systems.



\section{Conclusions}

The FIFO channel system model is used in many areas of the information
technology: from low-level hardware design to high-level software
specification and verification. Even though, in general, most decision
problems for the very expressive FIFO channel systems are undecidable,
there are many techniques based on model restrictions or/and by
using \emph{over-approximation} which make the systems appealing for
formal verifications.

In this paper we proposed a compositional semantics for FIFO channel
systems in terms of word relations. The semantics allows different
realizations of a system as long as the visible (external) aspect of
the system is preserved. Many practical aspects for which FIFO channel
systems were used in the past, such as the performance evaluation, are
not directly addressed by the semantics which says ``what'' not
``how''.  We believe however that the FIFO channel system model, which
is especially appealing because of its simple execution model, can
make the translation from ``what'' to ``how'' relatively
straightforward.

It is important to notice that the word relation semantics does not
consider any notion of time. In that asynchronous model we do not
distinguish between a delay and a deadlock. For example, a FIFO
component $C(0,1)={\downarrow}\{(1!a)\}$, i.e., which produces once a
signal $a$ on its unique output port, does not guarantee that the
output be ever produced. The observer (the user of the component) has
to wait for $a$ to be produced and, since the system is asynchronous,
the waiting time is not bounded.  In other words, the FIFO channel
system under the word relation semantics is a \emph{specification} of
a system, rather then its \emph{implementation}. In such a setting,
the problem of deadlock, for example, becomes a problem of performance
evaluation of a particular implementation.

\bibliography{fifo}
\bibliographystyle{plain}


\end{document}

